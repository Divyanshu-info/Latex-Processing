\section{Objective of Internship}

\subsection*{Training at Outlook}

We, during his internship at the Outlook Group, were appointed as an Knowledge Jockey in the Sales department. The role of a Knowledge Jockey (KJ) is assigned to interns who promote the offerings of Outlook Magazine while also having sales targets to achieve within a initial 10-day timeframe. The progression of my internship program unfolded as follows.

Right from the first day, we acquainted ourself with the Outlook sales team. As part of a group of students from ABV-IIITM, Gwalior, we were assigned in teams for the period of 10 days. I was under the guidance of Mr. Pritam Ghosh, who serves as a Human Resources (HR) professional in the Outlook Group.

As a Knowledge Jockey, my initial responsibilities involved promoting special online offers from Outlook and creating a comprehensive database of potential corporate and household customers. Subsequently, I would send these online offers to their respective email addresses. My role extended beyond promotion; I was also tasked with selling magazine subscriptions online. I was set a sales target of INR 7,000 to be achieved within a 10 days timeframe.

% \subsection{Task 1}
\subsection{Payment Gateway Analysis}

The assigned internship task involves a comprehensive exploration of payment gateways, their unique selling propositions (USPs), and a thorough review of selected payment gateways. Additionally, the task requires the formulation of a strategic analysis and recommendation for three preferred payment gateways, along with the preparation of a detailed Equated Monthly Installment (EMI) strategy for promoting higher value digital subscriptions. This research aims to provide insights into the payment gateway landscape and offer actionable strategies for enhancing subscription-based revenue streams.

\subsubsection{Task Breakdown}

\begin{enumerate}
    \item \textbf{Identifying Payment Gateway Companies:}
    The initial phase of the internship task involves conducting secondary research to identify and list 20 prominent payment gateway companies. This step is crucial for building a foundation to understand the landscape of available options.

    \item \textbf{Analyzing Payment Gateway USPs:}
    After identifying the payment gateway companies, the focus shifts to uncovering their unique selling propositions (USPs). This phase involves delving into the features, functionalities, and advantages that each payment gateway offers to clients and customers.

    \item \textbf{Reviewing Payment Gateways:}
    A subset of the identified payment gateways will be selected for a detailed review. For these 10 payment gateways, an in-depth examination of their descriptions, pros, cons, and top business tie-ups will be conducted. This review provides a holistic understanding of the strengths and limitations of each payment gateway.

    \item \textbf{Strategic Recommendation of Payment Gateways:}
    Based on the analysis conducted in the previous steps, the next stage is to recommend three payment gateways in order of preference. This recommendation will be supported by solid reasoning, aligning with the goals and requirements of the organization or project.

    \item \textbf{EMI Strategy for Digital Subscriptions:}
    The second part of the internship task involves formulating a detailed EMI strategy. This strategy is designed to effectively market and sell higher value digital subscriptions through well-structured Equated Monthly Installment plans. The strategy should outline pricing models, installment structures, customer benefits, and potential revenue projections.
\end{enumerate}

The internship task encompasses a multifaceted exploration of payment gateways, spanning from identifying the top players to analyzing their strengths and weaknesses. Additionally, it involves developing a strategic recommendation for optimal gateways and crafting an EMI strategy for boosting sales of premium digital subscriptions. This research effort aims to provide actionable insights and strategic directions to optimize revenue streams and enhance the customer experience in the realm of digital subscriptions.

% \subsection{Task 2}
\subsection{Security Market Analysis}

During our internship, Task 2 presented an exciting opportunity to delve into the intricacies of the financial world by undertaking a comprehensive analysis of the security market. This multifaceted task was divided into three core components, each requiring distinct skills and methodologies to accomplish.

\begin{enumerate}
    \item \textbf{Theoretical Analysis of Security Market:} The first aspect of Task 2 involved conducting a theoretical analysis of the security market. This encompassed a deep dive into the theoretical underpinnings of the financial market, including concepts like market efficiency, risk and return, portfolio diversification, and the role of various market participants such as investors, brokers, and regulators. By exploring these concepts, we aimed to gain a solid understanding of how the security market functions on a theoretical level and its implications for real-world investment decisions.

    \item \textbf{Performance Analysis of Top 10 Stocks (FY 2022-23):} The second facet of Task 2 required us to identify and analyze the performance of the top 10 stocks during the fiscal year 2022-23. This involved collecting and meticulously analyzing relevant financial data such as stock prices, trading volumes, earnings reports, and other financial indicators for each of the top 10 stocks. By conducting this performance analysis, we sought to discern trends, patterns, and factors that influenced the stock performance over the given period. Additionally, assessing the current situation of these stocks provided insights into their standing in the market at the time of the analysis.

    \item \textbf{Performance Analysis and Forecasting for Selected Stocks:} The final aspect of Task 2 centered around the in-depth analysis of a specific set of stocks: \textbf{Indusind Bank, Axis Bank, RBL, Motherson, Camlin Fine Sciences, Ramkrishna Forgings, HCC, Titan, and Havells}. For these stocks, we had to carry out a retrospective analysis of their performance over the previous years. This step involved evaluating historical stock prices, key financial ratios, market trends, and external influences that impacted their performance. Building on this analysis, the task extended to forecasting the performance of these stocks for the upcoming two years. This forecasting process demanded an amalgamation of financial acumen, analytical skills, and an understanding of the broader economic landscape.
\end{enumerate}

In conclusion, Task 2 of our internship provided an extensive platform to gain hands-on experience in the intricate realm of financial analysis. From exploring the theoretical foundations of the security market to dissecting the performance of top-performing stocks and projecting the trajectory of selected stocks, this task amalgamated theoretical knowledge with practical application. The insights gained from this task not only deepened our understanding of financial markets but also honed our analytical and forecasting capabilities, thereby contributing to a holistic learning experience during our internship journey.

% \subsection{Task 3}
\subsection{Investment Analysis and Portfolio Design}

During my internship, I was tasked with a comprehensive project that aimed to enhance my understanding of investment analysis and portfolio management. Task 3, in particular, was a multifaceted assignment that required me to delve into various aspects of investment, behavioral preferences, and portfolio design.
\begin{enumerate}

    \item  \textbf{Asset Class Identification and Analysis}

The first part of Task 3 involved identifying and analyzing 20 distinct asset classes where individuals could potentially invest. These asset classes spanned a range of investment options, such as stocks, mutual funds, real estate, bonds, commodities, and more. For each asset class, I prepared a descriptive chart that provided crucial information, including:

\begin{itemize}
    \item \textbf{Risk and Return Analysis:} I conducted a thorough assessment of the risk and return associated with each asset class. This involved examining historical data, volatility, and potential returns over time.
    \item \textbf{Benchmarking Agencies:} I researched and highlighted the benchmarking agencies that provide industry standards and performance comparisons for each asset class.
    \item \textbf{Top Performing Companies:} For asset classes like stocks and mutual funds, I identified and listed the top-performing companies or funds within each category.
\end{itemize}

\item  \textbf{Behavioral Aspects of Investment}

In the second part of the task, I conducted primary research to gain insights into the behavioral aspects of investment and investor preferences. This involved gathering responses from a minimum of 80 to a maximum of 200 individuals. The objective was to understand how psychological factors, risk tolerance, and emotional biases influence investment decisions. Through structured surveys or interviews, I aimed to uncover patterns and trends in investor behavior.

\item  \textbf{Portfolio Design}

The final component of Task 3 revolved around designing investment portfolios tailored to different earnings brackets. The task involved crafting logical investment strategies for individuals earning different annual incomes:

\begin{itemize}
    \item 10 Lakhs
    \item 15 Lakhs
    \item 20 Lakhs
    \item 30 Lakhs
    \item 50 Lakhs
    \item 1 Crore
\end{itemize}

The requirement was to design a minimum of 2 portfolios and a maximum of 6 portfolios for each earning class. The portfolio design was guided by the principles of diversification, risk tolerance, investment goals, and time horizon. Each portfolio was backed by logical reasoning, outlining why specific asset classes and investment instruments were chosen, as well as the intended risk-return profiles.
\end{enumerate}
Completing Task 3 required a blend of financial acumen, research skills, and an understanding of human psychology in investment decision-making. It provided a hands-on experience in identifying viable investment options, assessing risk and return, understanding investor behavior, and constructing investment portfolios aligned with varying income levels. This project not only deepened my knowledge in finance but also honed my ability to communicate complex financial concepts in a clear and concise manner.

\subsection{Analyzing Company Performance and Forecasting Returns}

During my internship, I was assigned Task 4, which proved to be an exciting and challenging opportunity. The objective of this task was to perform a comprehensive analysis of a company's financial health by closely examining its Balance Sheet and Profit and Loss (PnL) Statement. Additionally, I was required to forecast the company's potential returns over a two-year period. This task not only provided me with an in-depth understanding of financial analysis but also allowed me to develop forecasting skills that are vital in the world of business and finance.

To successfully complete this task, I followed a systematic approach:

\begin{enumerate}
    \item \textbf{Data Gathering:} I began by collecting the company's most recent Balance Sheet and PnL Statement. These documents provided valuable insights into the company's assets, liabilities, revenues, expenses, and profits over a specified period.
    
    \item \textbf{Financial Ratios and Metrics:} Using the data from the financial statements, I calculated a variety of financial ratios and metrics. These included liquidity ratios (like the current ratio and quick ratio), profitability ratios (such as gross profit margin and net profit margin), and solvency ratios (like debt-to-equity ratio). These ratios helped me assess different aspects of the company's financial performance and stability.
    
    \item \textbf{Trend Analysis:} By comparing financial data over multiple years, I was able to identify trends in the company's financial performance. This involved looking at how certain ratios and metrics evolved over time and understanding the implications of these trends.
    
    \item \textbf{Forecasting:} Forecasting the company's return over the next two years required a blend of financial expertise and market insights. I employed various forecasting techniques, such as trend analysis, time-series modeling, and industry growth projections. This step was crucial in predicting potential future outcomes based on historical data and market trends.
    
    \item \textbf{Risk Assessment:} To provide a comprehensive analysis, I also assessed potential risks that could impact the company's financial performance and subsequent returns. External factors, such as economic conditions, industry trends, and regulatory changes, were considered in this evaluation.
    
    \item \textbf{Report Generation:} After performing the analysis and forecasting, I compiled my findings into a comprehensive report. This report included an overview of the company, an analysis of its financial statements, a discussion of the forecasting methodologies used, and a presentation of the projected returns for the upcoming two years.
\end{enumerate}

In conclusion, Task 4 of my internship challenged me to delve deep into the financial aspects of a company and apply various analytical and forecasting techniques. This experience not only enhanced my understanding of financial analysis but also provided me with valuable insights into the practical application of these skills. Through this task, I developed the ability to interpret financial data, make informed predictions, and communicate complex financial concepts effectively.

\section{Literature Review}

With the increasing availability of internet access, the use of finance technologies in different sectors, particularly in the digital payment system, is booming. The Indian software market is expected to hit USD 100 billion in 2025 \Citep{Indianso78}, up from the present USD 12 billion. The growth of Fintech in India is illustrated by the increase in private equity and venture capital investments in the fintech sector from 2013 to 2022.

\citep{bhide2019growth} discusses the impact of digital payment gateways on various sectors in India, including e-commerce, transportation, and logistics. The use of digital payment gateways has led to increased transparency and efficiency in these sectors. The paper also highlights some of the digital payment methods available in India, including mobile wallets, UPI, and net banking. The government of India has launched several initiatives to promote the use of digital payment systems, including the Digital India campaign and the BHIM app. The paper concludes by emphasizing the need for continued investment in digital payment systems to support the growth of the Indian economy.

In the payments space, \Citep{durga2023study} discusses the rise of mobile payments and digital wallets, as well as the increasing adoption of Unified Payments Interface (UPI) and other payment platforms. The authors note that Fintech startups are driving innovation in this space by offering faster, more convenient, and more secure payment options to consumers. 

In the blockchain space, \citep{durga2023study} explores the potential of this technology to transform the financial services industry by enabling secure, transparent, and efficient transactions. The authors note that Fintech startups are leveraging blockchain to introduce new products and services, such as peer-to-peer lending platforms and digital identity solutions. 

\citep{patelpayment} highlights the different types of EMI, including closed-loop and open-loop systems. Closed-loop systems are used for specific purposes, such as gift cards or loyalty programs, while open-loop systems are used for general transactions, such as credit cards or mobile payments. 

\citep{patelpayment} discusses the regulatory framework for EMI, including the need for licensing and compliance with anti-money laundering (AML) and counter-terrorism financing (CTF) regulations. The paper also highlights the importance of customer trust in EMI, which can be achieved through transparency, security, and customer support. 

\Citep{bekkers2009strategic} explores the optimal asset allocation strategy for a portfolio consisting of ten different asset classes. The authors use a mean-variance analysis and a market portfolio approach to determine the optimal weights of each asset class in the portfolio. The ten asset classes considered in the study are global equities, global bonds, real estate, commodities, hedge funds, private equity, high-yield bonds, emerging market equities, emerging market debt, and inflation-linked bonds.

\Citep{shaik2022financial} indicate that IT professionals in India have a moderate level of financial literacy, with a majority of them having a basic understanding of financial concepts such as interest rates, inflation, and risk. The study also found that IT professionals in India prioritize their investment objectives based on their financial goals, risk appetite, and investment horizon. The most preferred investment options among IT professionals in India are mutual funds, followed by stocks and fixed deposits.

\citep{subbu2005multiobjective} present background on multiobjective optimization and the multiobjective portfolio optimization problem, and describe their hybrid evolutionary portfolio design architecture. They also present results from a highly constrained real-life portfolio design case with over fifteen hundred search dimensions. The results show that the hybrid approach is able to generate high-quality portfolios that satisfy all constraints and outperform traditional portfolio design methods. 

\Citep{sen2022stock} have conducted a study of various models in stock price prediction and portfolio optimization theories, including time series, machine learning, and deep learning models. The report's findings suggest that an efficient portfolio can be constructed by allocating different sector stocks optimally to achieve maximum return by taking minimum risk. The authors have also presented the optimal allocation of different sector stocks in the portfolio construction process. 

\Citep{sureshkumar2011efficient} that the artificial neural network (ANN) technique offers the most accurate predictions of stock prices, with a reduced error percentage compared to other methods. They also evaluate the performance of various functions, including the mean absolute percentage error (MAPE), mean absolute deviation (MAD), and root mean square error (RMSE). The results show that the ANN technique combined with the MAPE function offers the best performance in terms of accuracy and reduced error percentage. 

\Citep{tripathi2020performance} use various statistical tools such as Sharpe Ratio, Treynor Ratio, and Jensen's Alpha to measure the risk-adjusted returns of selected mutual fund schemes. The study finds that the selected mutual funds have performed well in terms of providing better returns to investors than benchmark indices. The average Sharpe Ratio of the funds is 5.81, indicating that they offer a better risk-adjusted return to investors. 

\Citep{choudhary2014performance} uses various financial tests to evaluate the performance of mutual funds, including Sharpe Ratio, Treynor Ratio, Jensen's Alpha, and Fama's Decomposition Analysis. The study concludes that the selected mutual funds have performed well in terms of returns and risk-adjusted returns. The study also provides a comparative analysis of the performance of different mutual funds and identifies the top-performing funds. 

% \usepackage{array}
% \usepackage{pdflscape}
% \usepackage{longtable}
% \usepackage{vcell}
% \pagenumbering{gobble}
\newgeometry{top=10mm, left=25mm, right=10mm, bottom=10mm}
\begin{small}
\begin{landscape}

    \begin{longtable}{|>{\hspace{0pt}}p{0.067\linewidth}|>{\hspace{0pt}}p{0.063\linewidth}|>{\hspace{0pt}}p{0.267\linewidth}|>{\hspace{0pt}}p{0.273\linewidth}|>{\hspace{0pt}}p{0.265\linewidth}|} 
    \hline
    \vcell{\textbf{Title}} & \vcell{\textbf{Author}} & \vcell{\textbf{DESCRIPTION}} & \vcell{\textbf{OUTCOME}} & \vcell{\textbf{RECOMMENDATION}} \\*[-\rowheight]
    \printcelltop & \printcelltop & \printcelltop & \printcelltop & \printcelltop \endfirsthead 
    \hline
    \vcell{Growth of Digital Payment System in India} & \vcell{\Citep{bhide2019growth}} & \vcell{This paper discusses the growth and impact of Fintech in digital payment gateways in India. It highlights the different digital payment methods available, the impact on various sectors, and the government's initiatives to promote digital payments. The study presented in the paper found a significant difference in the turnover of digital transactions executed through different modes of digital payments. The use of digital payment gateways has led to increased transparency and efficiency in various sectors.} & \vcell{The study presented in the paper found a significant difference in the turnover of digital transactions executed through different modes of digital payments. The use of digital payment gateways has led to increased transparency and efficiency in various sectors.} & \vcell{Continued investment in digital payment systems is necessary to support the growth of the Indian economy. The government should continue to promote the use of digital payments and encourage the development of new technologies.} \\*[-\rowheight]
    \printcelltop & \printcelltop & \printcelltop & \printcelltop & \printcelltop \\ 
    \hline
    \vcell{A Study on Fintech Start-Ups in India Special Refernce to Payments and Blockchain} & \vcell{\Citep{durga2023study}} & \vcell{This paper provides a conceptual study of Fintech startups in India, examining their impact on the financial services industry and the potential opportunities for growth and innovation. The study is based on secondary sources of information, including expert studies and official websites, and focuses on the areas of payments and blockchain.} & \vcell{The paper concludes that Fintech startups are well-positioned to shape the future of finance in India, with the potential to drive financial inclusion, revolutionize digital payments, and challenge traditional banking models. The authors note that Fintech startups are introducing innovative solutions in the areas of payments and blockchain, and that these solutions have the potential to transform the financial services industry by enabling faster, more convenient, and more secure transactions.} & \vcell{The authors recommend that policymakers and regulators in India continue to support the growth of Fintech startups, while also addressing the challenges and risks associated with this emerging industry. They suggest that Fintech startups should be encouraged to collaborate with traditional financial institutions, and that efforts should be made to promote financial literacy and consumer protection. Finally, the authors note that further research is needed to fully understand the impact of Fintech startups on the Indian financial services industry, and to identify the most effective strategies for promoting growth and innovation in this space.} \\*[-\rowheight]
    \printcelltop & \printcelltop & \printcelltop & \printcelltop & \printcelltop \\ 
    \hline
    \vcell{E-Payment – A Review on Different Methods, Issues and Challenges} & \vcell{\Citep{patelpayment}} & \vcell{The paper provides a comprehensive review of electronic payment systems, including the advantages and challenges of EMI (Electronic Money Institution). It discusses the different types of EMI, the regulatory framework for EMI, and the importance of customer trust in EMI. The paper also highlights the different organizational models used in e-payment systems and the challenges and opportunities that come with it.} & \vcell{The paper provides valuable insights into the future of electronic payment systems and the challenges and opportunities that come with it, particularly in the context of EMI. It highlights the importance of regulatory compliance and customer trust in the success of EMI, and provides a comprehensive overview of the different types of EMI and their uses. The paper also provides insights into the organizational models used in e-payment systems, which can help businesses and innovators build and implement mobile payment services that are more inclusive and accessible to all.} & \vcell{The paper is a valuable resource for anyone interested in the future of electronic payment systems and the challenges and opportunities that come with it, particularly in the context of EMI. It is recommended for entrepreneurs, innovators, policymakers, and researchers who are interested in the development and implementation of electronic payment systems. The paper provides a comprehensive overview of the current state of e-payment systems, highlighting the challenges and opportunities that come with it, and provides valuable insights into the organizational models used in e-payment systems.} \\*[-\rowheight]
    \printcelltop & \printcelltop & \printcelltop & \printcelltop & \printcelltop \\ 
    \hline
    \vcell{Strategic Asset Allocation: Determining the Optimal Portfolio with Ten Asset Classes} & \vcell{\Citep{bekkers2009strategic}} & \vcell{The paper explores the optimal asset allocation strategy for a portfolio consisting of ten different asset classes. The authors use a mean-variance analysis and a market portfolio approach to determine the optimal weights of each asset class in the portfolio. The ten asset classes considered in the study are global equities, global bonds, real estate, commodities, hedge funds, private equity, high-yield bonds, emerging market equities, emerging market debt, and inflation-linked bonds.} & \vcell{The authors find that adding non-traditional asset classes such as real estate, commodities, hedge funds, and private equity to a traditional asset mix of stocks, bonds, and cash can lead to a more optimal portfolio. They also find that the optimal portfolio weights for each asset class depend on the investor's risk tolerance and investment horizon.} & \vcell{The paper provides a useful framework for investors and financial advisors to determine the optimal asset allocation strategy for a diversified portfolio. The authors' approach is based on rigorous analysis and takes into account the unique risk and return characteristics of each asset class. Overall, the paper highlights the importance of diversification and the potential benefits of including non-traditional asset classes in a portfolio. Investors and financial advisors can use the findings of this paper to construct portfolios that are better suited to their investment goals and risk tolerance.} \\*[-\rowheight]
    \printcelltop & \printcelltop & \printcelltop & \printcelltop & \printcelltop \\ 
    \hline
    \vcell{Financial Literacy and Investment Behaviour of IT Professionals in India} & \vcell{\Citep{shaik2022financial}} & \vcell{This research paper explores the saving and investment behaviour of IT professionals in India, with a focus on identifying the factors that influence their investment behaviour, investment preferences, and risk aptitude. The study was conducted through a survey of 300 IT professionals in India, and the data was analyzed using descriptive statistics, factor analysis, and regression analysis.} & \vcell{The study found that IT professionals in India have a moderate level of financial literacy, with a majority of them having a basic understanding of financial concepts such as interest rates, inflation, and risk. The study also found that IT professionals in India prioritize their investment objectives based on their financial goals, risk appetite, and investment horizon. The most preferred investment options among IT professionals in India are mutual funds, followed by stocks and fixed deposits. The study also identified several challenges faced by IT professionals in India when it comes to investing their savings, such as lack of time, lack of knowledge, and lack of trust in financial institutions.} & \vcell{The study recommends that financial institutions and policymakers should take steps to improve financial literacy among IT professionals in India, provide them with easy-to-understand investment options, and build trust in the financial system. The study also highlights the need for financial education and awareness among this group. Overall, the study provides valuable insights into the investment behaviour of IT professionals in India and can be used to inform policies and strategies aimed at improving financial literacy and investment behaviour among this group.} \\*[-\rowheight]
    \printcelltop & \printcelltop & \printcelltop & \printcelltop & \printcelltop \\ 
    \hline
    \vcell{Multiobjective Financial Portfolio Design: A Hybrid Evolutionary Approach} & \vcell{\Citep{subbu2005multiobjective}} & \vcell{This paper presents a hybrid approach to multiobjective financial portfolio design that combines evolutionary computation with linear programming. The approach is designed to efficiently generate portfolios that satisfy all constraints, including risk and return objectives, diversification requirements, and transaction costs. The authors introduce a web-based environment that allows decision-makers to interactively down-select to a small subset of efficient portfolios via iterative constrained selections of portfolios represented as points projected in two-dimensional graphs over the combinations of the various return and risk measures utilized.} & \vcell{The authors demonstrate that their hybrid approach is a powerful tool for investment decision-making in industry, and has been successfully implemented at General Electric Asset Management, General Electric Insurance, and Genworth Financial. The approach is able to efficiently generate portfolios that satisfy all constraints, while promoting diversity via the injection of new points. The interactive graphical decision-making method allows decision-makers to quickly down-select to a small subset of efficient portfolios, making the approach practical for real-world applications.} & \vcell{This paper is recommended for researchers and practitioners in the field of computational finance who are interested in multiobjective financial portfolio design. The hybrid approach presented in this paper is a powerful tool for investment decision-making in industry, and has been successfully implemented at several large financial institutions. The interactive graphical decision-making method is particularly useful for decision-makers who need to quickly down-select to a small subset of efficient portfolios. Overall, this paper provides valuable insights into the design of multiobjective financial portfolios and demonstrates the effectiveness of the hybrid approach.} \\*[-\rowheight]
    \printcelltop & \printcelltop & \printcelltop & \printcelltop & \printcelltop \\ 
    \hline
    \vcell{An Efficient Approach to Forecasting Indian Stock Market Prices and their Performance Analysis} & \vcell{\Citep{sureshkumar2011efficient}} & \vcell{This paper explores various prediction algorithms and functions used to predict future share prices in the Indian stock market. The authors evaluate the accuracy of these models using historical data from the National Stock Exchange of India. They also compare the performance of various functions, including the mean absolute percentage error (MAPE), mean absolute deviation (MAD), and root mean square error (RMSE).} & \vcell{The authors find that the artificial neural network (ANN) technique offers the most accurate predictions of stock prices, with a reduced error percentage compared to other methods. They also find that the ANN technique combined with the MAPE function offers the best performance in terms of accuracy and reduced error percentage. The authors conclude that their approach provides a promising direction for the study of market predictions and their performance measures.} & \vcell{This paper provides valuable insights into the most effective techniques and functions for forecasting future share prices in the Indian stock market. The authors' approach and findings can be applied to other stock markets and investment contexts, providing a useful framework for future research efforts. The paper is recommended for investors, regulators, and researchers interested in stock market prediction and analysis.} \\*[-\rowheight]
    \printcelltop & \printcelltop & \printcelltop & \printcelltop & \printcelltop \\ 
    \hline
    \vcell{Stock Performance Evaluation: A Study of Indian Stock Market} & \vcell{\Citep{sen2022stock}} & \vcell{This report is a comprehensive analysis of the Indian stock market and the construction of an efficient portfolio to help investors earn maximum profit with minimum risk. The report covers various models in stock price prediction and portfolio optimization theories, including time series, machine learning, and deep learning models. The authors have conducted a study of the performance of these models in stock price prediction and portfolio optimization.} & \vcell{The report's findings suggest that an efficient portfolio can be constructed by allocating different sector stocks optimally to achieve maximum return by taking minimum risk. The authors have also presented the optimal allocation of different sector stocks in the portfolio construction process. The report provides valuable insights for investors interested in the Indian stock market and offers a foundation for further research in this area.} & \vcell{The authors recommend that investors interested in investing in the Indian stock market should consider constructing an efficient portfolio using the methods and models presented in this report. They also suggest that future research should focus on improving the accuracy of stock price prediction models and exploring new portfolio optimization theories.} \\*[-\rowheight]
    \printcelltop & \printcelltop & \printcelltop & \printcelltop & \printcelltop \\ 
    \hline
    \vcell{Performance Evaluation of Selected Equity Mutual Funds in India} & \vcell{\Citep{tripathi2020performance}} & \vcell{The paper provides a comprehensive analysis of the financial performance of open-end fund schemes in India's capital market. The study focuses on the risk-return relationship of equity mutual funds and evaluates their performance against benchmark indices. The authors use various statistical tools such as Sharpe Ratio, Treynor Ratio, and Jensen's Alpha to measure the risk-adjusted returns of selected mutual fund schemes.} & \vcell{The study finds that the selected mutual funds have performed well in terms of providing better returns to investors than benchmark indices. The average Sharpe Ratio of the funds is 5.81, indicating that they offer a better risk-adjusted return to investors. The study also highlights the differences in performance between large-cap, mid-cap, and small-cap equity mutual funds. The authors find that large-cap funds have lower risk and higher returns than mid-cap and small-cap funds. However, mid-cap and small-cap funds have the potential to generate higher returns in the long run.} & \vcell{The paper concludes by providing key factors that investors should consider when choosing profitable mutual funds for investment. These factors include the fund's investment objective, past performance, expense ratio, and risk profile. The study provides valuable insights for investors looking to invest in equity mutual funds in India's capital market. Investors should consider the findings of this study and conduct their own research before making investment decisions.} \\*[-\rowheight]
    \printcelltop & \printcelltop & \printcelltop & \printcelltop & \printcelltop \\ 
    \hline
    \vcell{Performance Evaluation of Mutual Funds: A Study of Selected Diversified Equity Mutual Funds in India} & \vcell{\Citep{choudhary2014performance}} & \vcell{This paper is a study on the performance evaluation of selected diversified equity mutual funds in India. The study aims to provide insights into the growth-oriented equity diversified schemes and their return and risk evaluation. The paper begins with an introduction to mutual funds, which are trusts that pool the savings of a number of investors who share a common financial goal. The money collected is then invested in capital market instruments such as shares, debentures, and other securities. The income earned through these investments and the capital appreciation realized is shared by its unit holders in proportion to the number of units owned by them.} & \vcell{The study uses various financial tests to evaluate the performance of mutual funds, including Sharpe Ratio, Treynor Ratio, Jensen's Alpha, and Fama's Decomposition Analysis. The study concludes that the selected mutual funds have performed well in terms of returns and risk-adjusted returns. The study also provides a comparative analysis of the performance of different mutual funds and identifies the top-performing funds.} & \vcell{The study provides valuable insights into the performance evaluation of mutual funds in India and can help investors make informed investment decisions. The study highlights the benefits of investing in a diversified basket of securities through mutual funds compared to direct stock market investment, as it offers an opportunity to invest in a professionally managed portfolio of securities at a relatively low cost. Investors can use the findings of this study to identify the top-performing mutual funds and make informed investment decisions based on their financial goals and risk appetite.} \\*[-\rowheight]
    \printcelltop & \printcelltop & \printcelltop & \printcelltop & \printcelltop \\
    \hline
    \end{longtable}
    \end{landscape}
\end{small}

\restoregeometry