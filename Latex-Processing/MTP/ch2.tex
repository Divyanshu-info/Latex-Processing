\chapter{Literature Review}
\label{mathchapter}
Naïve diversification, also known as equal-weighted portfolio construction, is a simple approach for reducing a portfolio's idiosyncratic risk without sacrificing the expected rate of return. It involves allocating equal weights to assets in the portfolio. In contrast, portfolio optimization, pioneered by Harry Markowitz, aims to find the optimal allocation of weights that achieves an acceptable expected return with minimal volatility.

Modern Portfolio Theory (MPT), by \citep{Markowitz1952}, maximizes returns for a given risk level via mean-variance portfolio construction. The theory centers on the efficient frontier, guiding investors in optimizing expected returns for a chosen risk level.

Markowitz's efficient frontier is a core MPT concept. It's the part of the minimum-variance curve extending above and to the right of the global minimum variance portfolio. Rational, risk-averse investors prefer portfolios on this frontier for their superior risk-return trade-off. As risk escalates, the efficient frontier curve flattens, underscoring a key MPT principle: continually pursuing higher returns entails disproportionately more risk. Thus, effective diversification is crucial to balancing risk and return. In essence, MPT encourages investors to construct portfolios along the efficient frontier to maximize expected returns while recognizing the diminishing returns associated with escalating risk.

Market capitalization-weighted portfolios have garnered criticism for their inefficiency and long-term underperformance compared to equally weighted portfolios. This observation is supported by studies such as those by \citep{Bolognesi201314} and \citep{Malladi2017188}. Additionally, research by \citep{DeMiguel20091915} not only reaffirmed the superior efficiency of equal-weighted portfolios over capitalization-weighted ones but also demonstrated that equal-weighting outperforms mean-variance-based portfolio strategies in out-of-sample testing.

Recent research, such as the study by \citep{Taljaard20211855} has highlighted a shift in the performance dynamics of equal-weighted portfolios. Specifically, it has been observed that an equal-weighted portfolio of stocks in the S\&P 500 now significantly underperforms the market capitalization-weighted portfolio, especially in the short term. This contrasts with earlier findings.

Moreover, \citep{Kritzman201031} argued that optimized portfolios, which are constructed through sophisticated optimization techniques, demonstrate superior out-of-sample performance when compared to equal-weighted portfolios. This suggests that optimization methods can provide better results than the traditional equal-weighting approach.

Given these evolving dynamics, it is prudent to consider a comprehensive comparison of various portfolio construction approaches, including both baseline methods like equal-weighting and more advanced optimization techniques. Such an assessment can provide a more nuanced understanding of the changing landscape of portfolio performance and help investors make informed decisions based on their specific objectives and investment horizons.

The foundation of the Markowitz efficient frontier is constructed upon certain assumptions that have faced scrutiny when applied to real-world contexts \citep{Ma2021}. Specifically, it presupposes that all investors exhibit rational behavior and are uniformly risk-averse. Furthermore, the model assumes that every investor enjoys equal access to borrowing funds at a risk-free interest rate, despite this not aligning with the actual circumstances. Additionally, the traditional concept of the efficient frontier operates on the premise that asset returns conform to a normal distribution, whereas, in practice, asset returns frequently deviate significantly, often extending as far as three standard deviations from the mean.

 Machine learning offers significant potential for the development of effective trading strategies, particularly in the realm of high-frequency trading, a feasibility that was previously limited \citep{Arnott201964}. The benefits of employing algorithmic machine learning in trading are extensive \citep{Zhang202025}, with a primary focus on enhancing alpha, or excess returns (\citep{Sirignano20191449} and \citep{Zhang20193001}). Much of the research concentrates on regression and classification pipelines, which involve forecasting excess returns or market movements over specific, predefined time horizons.

The utilization of machine learning techniques in finance has been on the rise, potentially influencing the way hedge fund managers assess risk-reward ratios within the financial market. Hedge funds, known for their adaptability, have been garnering increasing interest from investors \citep{Wu20218119}. In a comprehensive research endeavor, \citep{Wu20218119} employed machine learning for hedge fund return prediction and selection, demonstrating that machine learning-based forecasting methods consistently outperformed the respective Hedge Fund Research indices across various scenarios. Among the four machine learning methods examined by \citep{Wu20218119}, neural networks emerged as particularly noteworthy.

Concerning risk management, \citep{Arroyo2019124233} have demonstrated the utility of machine learning in aiding venture capital investors in their decision-making processes. This technology assists in identifying investment prospects and evaluating associated risks. Additionally, \citep{Jurczenko2020} has found that machine learning algorithms play a valuable role in enhancing stock risk forecasts, particularly when it comes to out-of-sample predictions of equity beta.

In a comprehensive research conducted by \citep{Gu20202223} it was found that machine learning tools surpass linear methods in terms of their predictive capabilities. The effectiveness of portfolios constructed using machine learning algorithms has been established, particularly for portfolios that have not undergone optimization, as highlighted by \citep{Kaczmarek20211}. Despite its widespread acceptance, the modern portfolio theory has faced criticism for its practical limitations (\citep{Kolm2014356} and \citep{DeMiguel20091915}). Consequently, a growing body of literature is dedicated to enhancing portfolio optimization techniques. This includes exploring alternatives such as replacing the statistical moments of asset returns with more reliable predictions \citep{DeMiguel20091915} or applying machine learning methods in place of traditional quadratic optimization, as proposed by \citep{DePrado201659}.

In the context of optimized strategies, \citep{Kaczmarek20211} have demonstrated that when machine learning methods are employed for the preselection of stocks within portfolios, conventional portfolio optimization methods such as mean-variance and hierarchical risk parity exhibit an enhancement in the risk-adjusted returns of these portfolios. This improvement results in the outperformance of equal-weighted portfolios in out-of-sample analyses.

Recently, there has been a growing interest in applying Deep learning techniques to portfolio allocation tasks within the field of finance. This surge in interest began following the seminal work of \citep{Zhang2019400}, leading to the emergence of a new branch of finance focused on the utilization of Deep learning methods in portfolio construction. Initially, these methods were applied to various financial domains, including cryptocurrencies, and as well as other asset classes.