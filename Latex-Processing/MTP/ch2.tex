\chapter{Literature Review}
\label{mathchapter}

\section{Types of Portfolio}


\cite{Gunjan2023} does categorization of portfolios based on the interplay of risk and return allows investors to tailor their investment strategies to align with their financial objectives and risk tolerance. This paper delineates five distinct portfolio categories, each catering to specific investment goals and preferences. Additionally, a behavioral perspective is introduced, expanding the classification into four types based on investor behavior and market dynamics.

\subsection{Portfolio Categories}

\subsubsection{Aggressive Portfolio}
\begin{itemize}
    \item \textbf{Objective:} Aims for higher returns.
    \item \textbf{Approach:} Willing to undertake higher risks, often favoring companies in initial growth stages.
    \item \textbf{Examples:} Companies with substantial growth potential.
\end{itemize}

\subsubsection{Defensive Portfolio}
\begin{itemize}
    \item \textbf{Objective:} Minimizes risk.
    \item \textbf{Approach:} Prefers companies offering daily need products, ensuring stability even in adverse conditions.
    \item \textbf{Examples:} Defensive industries providing essential goods and services.
\end{itemize}

\subsubsection{Income Portfolio}
\begin{itemize}
    \item \textbf{Objective:} Generates income from dividends or recurring benefits.
    \item \textbf{Approach:} Similar to a defensive portfolio but focuses on consistent returns.
    \item \textbf{Examples:} Real estate, FMCG, and stable industries providing regular dividends.
\end{itemize}

\subsubsection{Speculative Portfolio}
\begin{itemize}
    \item \textbf{Objective:} Pursues extremely high-risk opportunities.
    \item \textbf{Approach:} Often termed as gambling, involves investments in IPOs, initial product research, and takeover targets.
    \item \textbf{Examples:} Investments with high uncertainty and potential for significant gains or losses.
\end{itemize}

\subsubsection{Hybrid Portfolio}
\begin{itemize}
    \item \textbf{Objective:} Balances risk and return optimally.
    \item \textbf{Approach:} Utilizes a mix of different assets based on risk-return profiles.
    \item \textbf{Examples:} Diversified portfolios combining assets with varying degrees of risk.
\end{itemize}

\subsection{Behavioral Portfolio Categories}

\subsubsection{Transaction Cost Portfolio}
\begin{itemize}
    \item \textbf{Objective:} Reduces transaction costs during asset transactions.
    \item \textbf{Approach:} Utilizes time penalization techniques to minimize transaction fees and charges.
    \item \textbf{Examples:} Strategies focused on efficient buying and selling to minimize costs.
\end{itemize}

\subsubsection{Robust Portfolio}
\begin{itemize}
    \item \textbf{Objective:} Reduces transaction costs in a sparse and robust portfolio selection process.
    \item \textbf{Approach:} Mitigates estimation errors inherent in optimization processes.
    \item \textbf{Examples:} Strategies that enhance robustness in portfolio selection.
\end{itemize}

\subsubsection{Regularized Portfolio}
\begin{itemize}
    \item \textbf{Objective:} Addresses estimation errors in changing market conditions.
    \item \textbf{Approach:} Utilizes regularization techniques to handle uncertainties arising from market dynamics.
    \item \textbf{Examples:} Adaptive strategies that account for changing environments.
\end{itemize}

\subsubsection{Reinforcement Learning Portfolio}
\begin{itemize}
    \item \textbf{Objective:} Adapts to changing market conditions through continuous learning.
    \item \textbf{Approach:} Utilizes reinforcement learning techniques to dynamically adjust to evolving market dynamics.
    \item \textbf{Examples:} Strategies that incorporate machine learning to adapt to real-time market changes.
\end{itemize}

This comprehensive classification of portfolios based on risk-return dynamics and investor behavior provides a versatile framework for investors to tailor their investment strategies. By understanding the nuances of each portfolio type, investors can make informed decisions that align with their financial objectives and risk tolerance, promoting a more diversified and adaptive approach to portfolio management.


\section{The Indian Stock Market}
The Indian stock market has various distinguishing features compared to established financial markets like the New York Stock Exchange (NYSE). Despite being relatively small compared to the US market, India's stock market is expanding rapidly and presents substantial investment prospects. The market has experienced remarkable growth, with the number of listed companies increasing significantly from just a few initially to over 5,000 on the two leading stock exchanges, the Bombay Stock Exchange (BSE) and the National Stock Exchange of India (NSE), by 2020 \citep{Naik2020}.

To foster the development of the stock market, the Indian government has implemented reforms aimed at attracting foreign investors and enhancing transparency and accountability within the market. Experts anticipate that India's economy will continue to grow robustly in the forthcoming years, making it an appealing destination for investors seeking high returns. Presently, India's stock market is home to the world's fourth-largest equity market, following the United States, China, and Japan. The benchmark Sensex index, which monitors 30 prominent companies, has surged 10\% over the previous three months, while the broader Nifty 50 index has jumped 11\% during the same period \citep{Hiransha2018}.

Investors can gain exposure to the Indian stock market through the two primary stock exchanges: the Bombay Stock Exchange (BSE) and the National Stock Exchange (NSE). Both exchanges follow the same trading mechanism, trading hours, and settlement process. As of June 2023, the BSE had 5,657 listed firms, while the NSE had 2,137 as of March 31, 2023. Most significant Indian companies are listed on both exchanges \citep{Yadav2023}.

Foreign institutional investors, including mutual funds, pension funds, endowments, sovereign wealth funds, insurance companies, banks, and asset management companies, can invest directly in Indian stocks. High-net-worth individuals with a minimum net worth of \$50 million can register as sub-accounts of a Foreign Institutional Investor (FII). However, individual foreign investors cannot invest directly in the Indian stock market.

\citep{Surana2023} analysed that Indian stock market's liquidity employed multiple liquidity measures and utilized a Vector Auto-Regressive (VAR) approach. It found that the market displays interesting trading patterns among liquidity dimensions and emphasizes the significance of depth and tightness. The research revealed that the Indian stock market demonstrates greater persistence in depth, breadth, and immediacy but exhibits lower transaction costs compared to developed markets.

The Indian government has established rules allowing Foreign Institutional Investors (FIIs), Non-Resident Indians (NRIs), and Persons of Indian Origin (PIOs) to participate in the main and secondary capital markets via the Portfolio Investment Scheme (PIS). Through the stock exchanges in India, FIIs/NRIs are permitted to buy shares and debentures of Indian businesses. The corporation's paid-up capital determines the caps on overall FII and NRI/PIO investments. Public sector banks, including the State Bank of India, have a 20\% threshold. The 24\% FII investment ceiling can rise to the statutory or sectoral limit with authorization. Up to 24\%, the 10\% NRIs/PIOs ceiling may be raised. The reserve bank of India keeps track of these investments every day. \citep{rbi_foreign_investment}

\section{Baseline method and optimization portfolio construction}
Naïve diversification, also known as equal-weighted portfolio construction, is a simple approach for reducing a portfolio's idiosyncratic risk without sacrificing the expected rate of return. It involves allocating equal weights to assets in the portfolio. In contrast, portfolio optimization, pioneered by Harry Markowitz, aims to find the optimal allocation of weights that achieves an acceptable expected return with minimal volatility.

Modern Portfolio Theory (MPT), by \citep{Markowitz1952}, maximizes returns for a given risk level via mean-variance portfolio construction. The theory centers on the efficient frontier, guiding investors in optimizing expected returns for a chosen risk level.

Markowitz's efficient frontier is a core MPT concept. It's the part of the minimum-variance curve extending above and to the right of the global minimum variance portfolio. Rational, risk-averse investors prefer portfolios on this frontier for their superior risk-return trade-off. As risk escalates, the efficient frontier curve flattens, underscoring a key MPT principle: continually pursuing higher returns entails disproportionately more risk. Thus, effective diversification is crucial to balancing risk and return. In essence, MPT encourages investors to construct portfolios along the efficient frontier to maximize expected returns while recognizing the diminishing returns associated with escalating risk.

Market capitalization-weighted portfolios have garnered criticism for their inefficiency and long-term underperformance compared to equally weighted portfolios. This observation is supported by studies such as those by \citep{Bolognesi201314} and \citep{Malladi2017188}. Additionally, research by \citep{DeMiguel20091915} not only reaffirmed the superior efficiency of equal-weighted portfolios over capitalization-weighted ones but also demonstrated that equal-weighting outperforms mean-variance-based portfolio strategies in out-of-sample testing.

Recent research, such as the study by \citep{Taljaard20211855} has highlighted a shift in the performance dynamics of equal-weighted portfolios. Specifically, it has been observed that an equal-weighted portfolio of stocks in the S\&P 500 now significantly underperforms the market capitalization-weighted portfolio, especially in the short term. This contrasts with earlier findings.

Moreover, \citep{Kritzman201031} argued that optimized portfolios, which are constructed through sophisticated optimization techniques, demonstrate superior out-of-sample performance when compared to equal-weighted portfolios. This suggests that optimization methods can provide better results than the traditional equal-weighting approach.

Given these evolving dynamics, it is prudent to consider a comprehensive comparison of various portfolio construction approaches, including both baseline methods like equal-weighting and more advanced optimization techniques. Such an assessment can provide a more nuanced understanding of the changing landscape of portfolio performance and help investors make informed decisions based on their specific objectives and investment horizons.

\section{Machine Learning and optimization portfolio construction}
The foundation of the Markowitz efficient frontier is constructed upon certain assumptions that have faced scrutiny when applied to real-world contexts \citep{Ma2021}. Specifically, it presupposes that all investors exhibit rational behavior and are uniformly risk-averse. Furthermore, the model assumes that every investor enjoys equal access to borrowing funds at a risk-free interest rate, despite this not aligning with the actual circumstances. Additionally, the traditional concept of the efficient frontier operates on the premise that asset returns conform to a normal distribution, whereas, in practice, asset returns frequently deviate significantly, often extending as far as three standard deviations from the mean.

 Machine learning offers significant potential for the development of effective trading strategies, particularly in the realm of high-frequency trading, a feasibility that was previously limited \citep{Arnott201964}. The benefits of employing algorithmic machine learning in trading are extensive \citep{Zhang202025}, with a primary focus on enhancing alpha, or excess returns (\citep{Sirignano20191449} and \citep{Zhang20193001}). Much of the research concentrates on regression and classification pipelines, which involve forecasting excess returns or market movements over specific, predefined time horizons.

The utilization of machine learning techniques in finance has been on the rise, potentially influencing the way hedge fund managers assess risk-reward ratios within the financial market. Hedge funds, known for their adaptability, have been garnering increasing interest from investors \citep{Wu20218119}. In a comprehensive research endeavor, \citep{Wu20218119} employed machine learning for hedge fund return prediction and selection, demonstrating that machine learning-based forecasting methods consistently outperformed the respective Hedge Fund Research indices across various scenarios. Among the four machine learning methods examined by \citep{Wu20218119}, neural networks emerged as particularly noteworthy.

Concerning risk management, \citep{Arroyo2019124233} have demonstrated the utility of machine learning in aiding venture capital investors in their decision-making processes. This technology assists in identifying investment prospects and evaluating associated risks. Additionally, \citep{Jurczenko2020} has found that machine learning algorithms play a valuable role in enhancing stock risk forecasts, particularly when it comes to out-of-sample predictions of equity beta.

In a comprehensive research conducted by \citep{Gu20202223} it was found that machine learning tools surpass linear methods in terms of their predictive capabilities. The effectiveness of portfolios constructed using machine learning algorithms has been established, particularly for portfolios that have not undergone optimization, as highlighted by \citep{Kaczmarek20211}. Despite its widespread acceptance, the modern portfolio theory has faced criticism for its practical limitations (\citep{Kolm2014356} and \citep{DeMiguel20091915}). Consequently, a growing body of literature is dedicated to enhancing portfolio optimization techniques. This includes exploring alternatives such as replacing the statistical moments of asset returns with more reliable predictions \citep{DeMiguel20091915} or applying machine learning methods in place of traditional quadratic optimization, as proposed by \citep{DePrado201659}.

Reinforcement learning has found application in the domain of algorithmic trading, as highlighted by the work of Wen et al. (2021). Their research in options trading demonstrates the effectiveness of a reinforcement learning model, which outperforms the traditional buy-and-hold strategy in generating favorable returns. In the context of futures contracts, the study conducted by Zhang et al. (2020) establishes the superiority of trading strategies employing reinforcement learning over time series momentum strategies, yielding positive profits even in the presence of considerable transaction costs. Additionally, Kolm and Ritter (2019) explore the use of reinforcement learning to train a machine learning algorithm for options hedging in realistic scenarios, specifically identifying the minimal variance hedge based on a given transaction cost function.

In the realm of reinforcement learning, an agent interacts with its environment, learning an optimal course of action through trial and error, utilizing reward and penalty points for successful and erroneous actions, respectively. This approach is particularly relevant for problems involving sequential decision-making (Li, 2017). Unlike deep learning models, which rely on extensive historical datasets, reinforcement learning stands out due to its capacity for self-learning and adaptation based on the environment. In dynamically changing investment landscapes, where volatility is constant, the adaptability of reinforcement learning makes it a more attractive option compared to the reliance on historical datasets in deep learning. Successful integration of reinforcement learning in asset management and portfolio construction could potentially reshape the risk and return standards within the financial sector. However, empirical evidence supporting the application of reinforcement learning models in asset management remains limited as of now.

In the context of optimized strategies, \citep{Kaczmarek20211} have demonstrated that when machine learning methods are employed for the preselection of stocks within portfolios, conventional portfolio optimization methods such as mean-variance and hierarchical risk parity exhibit an enhancement in the risk-adjusted returns of these portfolios. This improvement results in the outperformance of equal-weighted portfolios in out-of-sample analyses.

\section{Deep Learning and optimization portfolio construction}

Recently, there has been a growing interest in applying Deep learning techniques to portfolio allocation tasks within the field of finance. This surge in interest began following the seminal work of \citep{Zhang2019400}, leading to the emergence of a new branch of finance focused on the utilization of Deep learning methods in portfolio construction. Initially, these methods were applied to various financial domains, including cryptocurrencies, and as well as other asset classes.

Researchers have conducted comprehensive investigations to address the challenge of time series forecasting for stock returns using deep learning techniques (\citep{Chiang2016}, \citep{DiPersio2016}, \citep{Moghaddam2016}). Several studies propose that various Recurrent Neural Networks (RNN) exhibit superior performance compared to conventional financial time series models across diverse markets (\cite{Bao2017}, \cite{Chen2015}, 4\cite{Sermpinis2019}). The recurrent layer in RNN incorporates feedback loops, facilitating the retention of information in a "memory cell" over time. Nevertheless, its effectiveness diminishes when confronted with learning tasks that involve extended long-term temporal dependencies.

Long Short-Term Memory (LSTM) represents a distinctive category of Recurrent Neural Networks (RNN) that has demonstrated efficacy in text mining for forecasting stock returns \cite{Kraus2017}. LSTM incorporates specialized "memory cells" capable of preserving information over extended time intervals \cite{Hochreiter19971735}. As a result, LSTM frequently exhibits superior performance in handling sequential data and making predictions in financial time series, surpassing the capabilities of traditional RNN models (\cite{Jiang2019}, \cite{Nelson2017}). This advantage is particularly pronounced in the SRI context, where investors prioritize long-term returns over the short-term market's volatility.

Scholars have undertaken comparisons of various RNN architectures, including Long Short-Term Memory (LSTM) and Gated Recurrent Unit (GRU) networks \cite{Samarawickrama2017}. Some researchers have proposed that the Bi-directional LSTM (BiLSTM) could present a more favorable alternative for similar Optimization problems \cite{Chen2017}. While LSTM and GRU, featuring a unidirectional flow of information, may suffice for many sequence prediction tasks, the BiLSTM model reviews the data in both forward and backward directions \cite{Schuster1997}. This dual-directional approach contributes to enhanced optimization accuracy, especially in the optimization of sequential data such as financial time series.

