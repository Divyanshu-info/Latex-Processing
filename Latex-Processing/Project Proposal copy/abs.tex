\section*{Executive Summary}
Patanjali Ayurved Limited is proposing to set up an additional manufacturing plant in Surat, Gujarat. The new plant will help Patanjali to meet the growing demand for its products, reduce production costs, improve product quality, and create employment opportunities in Surat and Gujarat.

The project is expected to cost Rs. 125 crore and will be funded by a combination of equity and debt. Patanjali will invest Rs. 50 crore in equity and raise Rs. 75 crore in debt from commercial banks or non-banking financial companies.

The project is expected to be completed in 18 months. The new plant is expected to commence commercial production in 12 months.

The project is expected to generate a number of significant benefits, including:

\begin{itemize}
    \item Increased production capacity: The new plant will increase Patanjali's production capacity by 20\%.
    \item Reduced production costs: The lower cost of land, labor, and electricity in Gujarat will help Patanjali to reduce its production costs by 10\%.
    \item Improved product quality: The new plant will be equipped with state-of-the-art machinery and equipment, which will help Patanjali to improve the quality of its products.
    \item Employment generation: The new plant is expected to create over 1,000 direct and indirect jobs.
    \item Economic growth: The project is expected to boost the economy of Surat and Gujarat.
\end{itemize}

The project is also expected to have a number of social and environmental benefits, including:

\begin{itemize}
    \item Promotion of ayurvedic medicine: The project will help to promote ayurvedic medicine in India and abroad.
    \item Support for farmers: The project will source raw materials from Indian farmers, which will support their livelihoods.
    \item Conduct of research and development: The project will invest in research and development to develop new ayurvedic products and improve the quality of existing products.
    \item Environmental protection: The project will adopt sustainable practices to minimize its environmental impact.
\end{itemize}

Overall, the project to set up an additional manufacturing plant of Patanjali in Surat, Gujarat is a viable and beneficial project. The project is expected to generate a number of significant economic, social, and environmental benefits.

