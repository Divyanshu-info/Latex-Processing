\section{Analyze the performance of below stocks and forecast the performance for next 2 years}
\subsection{Indusind Bank}
\begin{enumerate}
    \item In the fourth quarter of FY23, the Net Interest Income (NII) experienced a growth of 3.9\% quarter-on-quarter (QoQ) and 17.2\% year-on-year (YoY), reaching INR 46,695 million. This growth indicates a positive trend in the bank's interest-earning activities. The Net Interest Margins (NIMs) for Q4FY23 were 4.28\%, which represents an expansion of 8 basis points (bps) YoY and 1 bps QoQ. This expansion in NIMs suggests an improvement in the bank's interest spread and efficiency in managing its interest-bearing assets and liabilities.

\item For the entire fiscal year 2023 (FY23), the NII amounted to INR 175,921 million, exhibiting a YoY growth of 17.3\%. The margins for the full year were reported at 4.3\%, indicating a consistent performance in interest income generation.

\item The Pre-Provision Operating Profits (PPOP) witnessed a YoY growth of 11.1\% and a QoQ growth of 1.9\% in Q4FY23, reaching INR 37,575 million. PPOP for the entire fiscal year amounted to INR 144,190 million, reflecting a YoY growth of 10.6\%. This growth in PPOP indicates the bank's ability to generate profits from its core operations before accounting for provisions for loan losses and other contingencies.

\item The net profit for Q4FY23 stood at INR 20,434 million, representing a significant YoY growth of 45.9\% and a modest QoQ growth of 4.1\%. The net profit for the full fiscal year 2023 grew by 54.9\% YoY, amounting to INR 74,431 million. These figures highlight the bank's improved profitability and efficiency in managing its expenses and credit quality.

\item As of March 31, 2023, the Gross Non-Performing Assets (GNPA) and Net Non-Performing Assets (NNPA) ratios were reported at 1.98\% and 0.59\%, respectively. These figures demonstrate an improvement compared to the GNPA of 2.06\% and NNPA of 0.62\% reported as of December 31, 2022. The declining NPA ratios indicate better asset quality and a lower proportion of non-performing loans in the bank's loan portfolio.

\item The bank experienced a YoY growth of 21.3\% and a QoQ growth of 6.3\% in advances, which reached INR 28,99,237 million as of March 31, 2023. This growth in advances suggests an increase in the bank's lending activities and its ability to attract borrowers. On the other hand, deposits grew by 14.6\% YoY and 3.3\% QoQ, amounting to INR 33,61,202 million as of the same date. The growth in deposits indicates the bank's ability to gather funds from customers and maintain a stable funding base.

\item In terms of capital adequacy, the bank's total Capital Adequacy Ratio (CAR) as per Basel III guidelines was reported at 17.86\% as of March 31, 2023. Additionally, the Tier 1 Capital to Risk-Weighted Assets Ratio (CRAR) stood at 16.37\%. These ratios indicate that the bank maintains a strong capital position, complying with regulatory requirements and having a sufficient cushion to absorb potential losses.


\end{enumerate}
 Overall, the bank showcased positive financial performance in various aspects, including net interest income, profitability, asset quality, loan growth, deposit growth, and capital adequacy. These results signify the bank's resilience and sound financial management in a challenging economic environment. The bank's net interest income is expected to grow at a compound annual growth rate (CAGR) of 8\% over FY23-25E. This suggests that the bank's lending activities are expected to generate a consistent income in the coming years.

\subsubsection{Key Financial}
% \usepackage{tabularray}
\begin{longtblr}[
  caption = {Profit \& Loss Statement},
]{
  width = \linewidth,
  colspec = {Q[333]Q[119]Q[119]Q[119]Q[119]Q[119]},
  hline{1,16} = {-}{0.08em},
  hline{2} = {-}{0.05em},
}
INR Mn & FY21 & FY22 & FY23 & FY24E & FY25E\\
Interest Income & 2,89,998 & 3,08,224 & 3,63,679 & 4,47,508 & 5,34,398\\
Interest Expense & 1,54,719 & 1,58,216 & 1,87,758 & 2,40,551 & 2,78,549\\
Net Interest Income & 1,35,279 & 1,50,008 & 1,75,921 & 2,06,957 & 2,55,849\\
Non interest income & 65,586 & 73,971 & 81,728 & 96,439 & 1,13,798\\
Operating income & 2,00,865 & 2,23,979 & 2,57,649 & 3,03,396 & 3,69,646\\
- Employee expense & 22,135 & 24,883 & 41,787 & 52,179 & 63,424\\
- Other operating expense & 61,463 & 70,710 & 71,672 & 81,917 & 97,956\\
Operating Expense & 83,598 & 95,593 & 1,13,459 & 1,34,096 & 1,61,380\\
PPOP & 1,17,267 & 1,28,386 & 1,44,190 & 1,69,300 & 2,08,266\\
Provisions & 79,425 & 66,650 & 44,868 & 45,596 & 51,317\\
PBT & 37,841 & 61,737 & 99,322 & 1,23,704 & 1,56,950\\
Tax Expense & 9,478 & 15,625 & 24,887 & 30,926 & 39,237\\
PAT & 28,364 & 46,111 & 74,435 & 92,778 & 1,17,712\\
Diluted EPS (INR) & 38.8 & 59.5 & 96.0 & 119.6 & 151.7
\end{longtblr}


% \usepackage{tabularray}
\begin{longtblr}[
  caption = {Balance Sheet},
]{
  width = \linewidth,
  colspec = {Q[304]Q[127]Q[127]Q[127]Q[127]Q[127]},
  hline{1,18} = {-}{0.08em},
  hline{2} = {-}{0.05em},
}
INR  Mn & FY21 & FY22 & FY23 & FY24E & FY25E\\
Source of Funds & NaN & NaN & NaN & NaN & NaN\\
Share capital & 7,734 & 7,747 & 7,759 & 7,759 & 7,759\\
Reserves  Surplus & 4,25,866 & 4,69,065 & 5,41,844 & 6,16,279 & 7,09,057\\
Networth & 4,33,600 & 4,76,812 & 5,49,603 & 6,24,038 & 7,16,816\\
ESOP & 54 & 161 & 443 & 443 & 443\\
Borrowings & 5,13,228 & 4,73,232 & 4,90,112 & 5,50,229 & 5,78,081\\
Deposits & 25,62,050 & 29,36,814 & 33,61,202 & 39,30,210 & 46,24,649\\
Other liabilities  provisions & 1,20,796 & 1,32,728 & 1,77,006 & 2,59,154 & 3,79,428\\
Total Equity  Liabilities & 36,29,728 & 40,19,746 & 45,78,366 & 53,64,075 & 62,99,417\\
Uses of Funds & NaN & NaN & NaN & NaN & NaN\\
Balances w/ banks  others & 5,63,272 & 6,82,745 & 1,38,019 & 5,10,927 & 6,01,204\\
Investments & 6,96,947 & 7,09,708 & 8,30,757 & 9,82,553 & 11,56,162\\
Loans  advances & 21,25,954 & 23,90,515 & 28,99,237 & 34,32,485 & 41,20,070\\
Fixed assets & 18,094 & 18,487 & 20,789 & 23,908 & 27,494\\
Other assets & 2,25,461 & 2,18,291 & 2,59,816 & 2,37,343 & 1,86,377\\
Total Assets & 36,29,728 & 40,19,746 & 45,78,366 & 53,64,075 & 62,99,417
\end{longtblr}

% \usepackage{tabularray}
\begin{longtblr}[
  caption = {Ratio Analysis},
]{
  width = \linewidth,
  colspec = {Q[438]Q[104]Q[94]Q[94]Q[100]Q[100]},
  hline{1,35} = {-}{0.08em},
  hline{2} = {-}{0.05em},
}
Key Ratio & FY21 & FY22 & FY23 & FY24E & FY25E\\
Growth Rates & NaN & NaN & NaN & NaN & NaN\\
Advances (\%) & 2.8\% & 12.4\% & 21.3\% & 18.4\% & 20.0\%\\
Deposits (\%) & 26.8\% & 14.6\% & 14.6\% & 16.9\% & 17.7\%\\
Total assets (\%) & 18.2\% & 10.7\% & 13.9\% & 17.2\% & 17.4\%\\
NII (\%) & 12.2\% & 10.9\% & 17.3\% & 17.6\% & 23.6\%\\
Pre-provisioning profit (\%) & 8.9\% & 9.5\% & 10.6\% & 17.4\% & 23.0\%\\
PAT (\%) & -35.8\% & 62.6\% & 54.9\% & 24.6\% & 26.9\%\\
B/S Ratios & NaN & NaN & NaN & NaN & NaN\\
Credit/Deposit (\%) & 83.0\% & 81.4\% & 86.3\% & 87.3\% & 89.1\%\\
CASA (\%) & 41.8\% & 38.7\% & 40.1\% & 32.8\% & 29.7\%\\
Advances/Total assets (\%) & 58.6\% & 59.5\% & 63.3\% & 64.0\% & 65.4\%\\
Leverage - Total Assets to Equity & 8.37 & 8.43 & 8.33 & 8.60 & 8.79\\
Operating efficiency & NaN & NaN & NaN & NaN & NaN\\
Cost/income (\%) & 41.6\% & 42.7\% & 44.0\% & 44.2\% & 43.7\%\\
Opex/total assets (\%) & 2.7\% & 2.4\% & 2.5\% & 2.5\% & 2.6\%\\
Opex/total interest earning assets & 3.0\% & 3.2\% & 3.3\% & 3.3\% & 3.3\%\\
Profitability & NaN & NaN & NaN & NaN & NaN\\
NIM (\%) & 4.6\% & 4.4\% & 5.0\% & 4.7\% & 4.7\%\\
RoA (\%) & 0.8\% & 1.1\% & 1.6\% & 1.7\% & 1.9\%\\
RoE (\%) & 6.5\% & 9.7\% & 13.5\% & 14.9\% & 16.4\%\\
Asset quality & NaN & NaN & NaN & NaN & NaN\\
Gross NPA (\%) & 2.7\% & 2.3\% & 2.0\% & 2.0\% & 1.9\%\\
Net NPA (\%) & 0.7\% & 0.6\% & 0.6\% & 0.6\% & 0.6\%\\
PCR (\%) & 75.0\% & 71.7\% & 70.6\% & 71.0\% & 71.0\%\\
Slippage (\%) & 1.9\% & 1.0\% & 1.0\% & 1.0\% & 1.0\%\\
Credit cost (\%) & 2.9\% & 2.3\% & 1.3\% & 1.3\% & 1.3\%\\
Per share data / Valuation & NaN & NaN & NaN & NaN & NaN\\
EPS (INR) & 36.7 & 59.5 & 95.9 & 119.6 & 151.7\\
BVPS (INR) & 560.7 & 615.5 & 708.3 & 804.3 & 923.9\\
ABVPS (INR) & 527.9 & 595.8 & 686.2 & 778.8 & 893.5\\
P/E (x) & 32.5 & 20.0 & 11.7 & 9.4 & 7.4\\
P/BV (x) & 2.1 & 1.9 & 1.6 & 1.4 & 1.2\\
P/ABV (x) & 2.3 & 2.0 & 1.6 & 1.4 & 1.3
\end{longtblr}

\subsection{Axis Bank}
\begin{enumerate}
    \item The consolidated ROE (excluding exceptional items) for FY23 was 19\%, led by all-round outperformance across NIMs, fees, costs, and asset quality metrics. Axis Bank Ltd (AXSB) delivered a growth of 68\% in Profit After Tax (excluding exceptional items) driven by a 30.0\% growth in net interest income and 25\% growth in fee income. The NIMs for the entire year improved by 55 bps YoY to 4.02\%, while credit costs declined by 32 bps YoY to 0.40\%.
    
    \item The focus on building a quality and granular liability franchise resulted in a 21\% YoY growth in granular CASA deposits. AXSB achieved an 870 bps YoY increase in the share of the premium segment in the Retail SA deposits portfolio and a 550 bps YoY reduction in the overall deposits' outflow rates.
    
    \item AXSB continued to grow faster than the industry with a domestic loan book growth of 23\% YoY. The focus segments comprising Mid Corporate, SME, and Small Business Banking (SBB) grew at a much higher pace of 32\% YoY and constituted 20\% of the overall loan book in FY23.
    
    \item The bank's retail advances book grew 22\%, aided by several large transformation and technology initiatives. AXSB strengthened its market share in retail cards and payments businesses through innovative product propositions and partnerships driven by the Known to Bank (KTB) strategy.
    
    \item AXSB's positioning in the credit cards business improved, with its card advances market share increasing by 450 bps to 16.3\%, supported by the acquisition of a quality and complementary credit card franchise. The liability franchise received a boost with over 100 bps improvement in CASA ratio and access to over 1,600 Suvidha corporate relationships through integration.
    
    \item The acquired CITI business, while running at a higher cost, is ROE accretive post-integration and strengthens AXSB's market presence in the retail space. The integration between the two institutions enables AXSB to create the gold standard in the retail segment.
    
    \item The MSME segment remains a key growth driver for the bank, with the combined loan portfolio of Mid Corporate and SME doubling in the last three years. This has brought higher granularity and supported the Priority Sector Lending (PSL) agenda of the bank.
    
    \item Axis Finance achieved a 30\% YoY growth in net profit with an ROE of 16.9\%, a capital adequacy ratio of over 20\%, and superior asset quality. Axis AMC delivered a PAT growth of 16\% YoY, while the retail brokerage subsidiary delivered a PAT of INR 2,030 Mn despite a volatile market environment. Axis Capital continued to maintain its dominance in equity capital markets.
    
    \item The Board recommended a dividend of INR 1 per equity share of face value INR 2 each for FY23, considering the bank's overall performance and capital retention for future growth.
\end{enumerate}
\subsubsection{Key Financials}

% \usepackage{tabularray}
\begin{longtblr}[
  caption = {Profit \& Loss Statement},
]{
  width = \linewidth,
  colspec = {Q[325]Q[115]Q[115]Q[115]Q[131]Q[129]},
  hline{1,17} = {-}{0.08em},
  hline{2} = {-}{0.05em},
}
INR Mn & FY 21 & FY 22 & FY 23 & FY 24E & FY 25E\\
Interest Income & 6,36,453 & 6,73,768 & 8,51,638 & 10,63,599 & 12,45,211\\
Interest Expense & 3,44,062 & 3,42,446 & 4,22,180 & 5,41,168 & 6,33,312\\
Net Interest Income & 2,92,391 & 3,31,322 & 4,29,458 & 5,22,432 & 6,11,899\\
Non-interest income & 1,48,382 & 1,52,205 & 1,65,009 & 1,91,938 & 2,26,487\\
Operating income & 4,40,773 & 4,83,528 & 5,94,466 & 7,14,369 & 8,38,386\\
- Employee expense & 61,640 & 76,126 & 87,974 & 93,233 & 1,02,376\\
- Other operating expense & 1,22,111 & 1,59,982 & 1,86,009 & 2,28,598 & 2,64,092\\
Operating Expense & 1,83,752 & 2,36,108 & 2,73,983 & 3,21,832 & 3,66,467\\
PPOP & 2,57,022 & 2,47,420 & 3,20,483 & 3,92,538 & 4,71,919\\
Provisions & 1,68,963 & 73,595 & 26,526 & 47,019 & 60,977\\
PBT & 88,058 & 1,73,826 & 2,93,957 & 3,45,519 & 4,10,941\\
Tax Expense & 22,173 & 43,571 & 73,262 & 86,380 & 1,02,735\\
Exceptional Expenses & 0 & 0 & 1,24,898 & 0 & 0\\
PAT & 65,885 & 1,30,255 & 95,797 & 2,59,139 & 3,08,206\\
Diluted EPS (INR) & 21.5 & 42.5 & 31.1 & 84.2 & 100.2
\end{longtblr}



% \usepackage{tabularray}
\begin{longtblr}[
  caption = {Balance Sheet},
]{
  width = \linewidth,
  colspec = {Q[210]Q[144]Q[144]Q[144]Q[144]Q[144]},
  hline{1,22} = {-}{0.08em},
  hline{2} = {-}{0.05em},
}
INR  Mn & FY 21 & FY 22 & FY 23 & FY 24E & FY 25E\\
Source of Funds & NaN & NaN & NaN & NaN & NaN\\
Share capital & 6,120 & 6,140 & 6,154 & 6,154 & 6,154\\
Reserves  Surplus & 10,09,225 & 11,45,601 & 12,48,013 & 15,07,152 & 18,15,358\\
Networth & 10,15,345 & 11,51,741 & 12,54,167 & 15,13,305 & 18,21,511\\
Borrowings & 14,28,732 & 18,51,339 & 18,63,000 & 21,46,650 & 24,51,765\\
Deposits & 69,79,853 & 82,19,716 & 94,69,452 & 1,10,08,464 & 1,29,04,024\\
Other liabilities &  &  &  &  & \\
provisions & 4,44,051 & 5,31,493 & 5,86,636 & 9,08,330 & 9,71,239\\
Total Equity &  &  &  &  & \\
Liabilities & 98,67,981 & 1,17,54,288 & 1,31,73,255 & 1,55,76,750 & 1,81,48,540\\
Uses of Funds & NaN & NaN & NaN & NaN & NaN\\
Cash &  &  &  &  & \\
Balance with RBI & 6,17,303 & 20,50,212 & 17,25,286 & 21,04,023 & 22,32,138\\
Other Bank and &  &  &  &  & \\
Call Money & 22,61,196 & 27,55,972 & 28,88,148 & 33,57,581 & 40,00,248\\
Net investments & 61,43,994 & 70,79,466 & 84,53,028 & 99,94,191 & 1,17,79,977\\
Loans  advances & 42,450 & 45,724 & 47,339 & 50,339 & 53,339\\
Fixed assets & 8,03,038 & 7,63,257 & 7,20,632 & 7,64,853 & 8,11,787\\
Other assets & 98,67,981 & 1,17,54,288 & 1,31,73,255 & 1,55,76,750 & 1,81,48,540\\
Total Assets & 4,53,44,296 & 4,98,75,974 & 5,51,69,785 & 6,18,29,896 & 6,89,86,107
\end{longtblr}

% \usepackage{tabularray}
\begin{longtblr}[
  caption = {Ratio Analysis},
]{
  width = \linewidth,
  colspec = {Q[425]Q[106]Q[90]Q[100]Q[106]Q[104]},
  hline{1,35} = {-}{0.08em},
  hline{2} = {-}{0.05em},
}
Key Ratio & FY 21 & FY 22 & FY 23 & FY 24E & FY 25E\\
Growth Rates & NaN & NaN & NaN & NaN & NaN\\
Advances (\%) & 7.5\% & 15.2\% & 19.4\% & 18.2\% & 17.9\%\\
Deposits (\%) & 9.0\% & 17.8\% & 15.2\% & 16.3\% & 17.2\%\\
Total assets (\%) & 7.8\% & 19.1\% & 12.1\% & 18.2\% & 16.5\%\\
NII (\%) & 16.0\% & 13.3\% & 29.6\% & 21.6\% & 17.1\%\\
Pre-provisioning profit (\%) & 9.7\% & -3.7\% & 29.5\% & 22.5\% & 20.2\%\\
PAT (\%) & 304.9\% & 97.7\% & -26.5\% & 170.5\% & 18.9\%\\
B/S Ratios & NaN & NaN & NaN & NaN & NaN\\
Credit/Deposit (\%) & 88.0\% & 86.1\% & 89.3\% & 90.8\% & 91.3\%\\
CASA (\%) & 38.5\% & 45.0\% & 47.2\% & 47.3\% & 47.4\%\\
Advances/Total assets (\%) & 62.3\% & 60.2\% & 64.2\% & 64.2\% & 64.9\%\\
Leverage - Total Assets to Equity & 9.7 & 10.2 & 10.5 & 10.3 & 10.0\\
Operating efficiency & NaN & NaN & NaN & NaN & NaN\\
Cost/income (\%) & 41.7\% & 48.8\% & 46.1\% & 45.1\% & 43.7\%\\
Opex/total assets (\%) & 1.9\% & 2.0\% & 2.1\% & 2.1\% & 2.0\%\\
Opex/total interest earning assets & 2.2\% & 2.4\% & 2.4\% & 2.4\% & 2.3\%\\
Profitability & NaN & NaN & NaN & NaN & NaN\\
NIM (\%) & 3.7\% & 3.5\% & 3.8\% & 4.0\% & 4.0\%\\
RoA (\%) & 0.7\% & 1.2\% & 0.8\% & 1.8\% & 1.8\%\\
RoE (\%) & 7.1\% & 12.0\% & 8.0\% & 18.7\% & 18.5\%\\
Asset quality & NaN & NaN & NaN & NaN & NaN\\
Gross NPA (\%) & 3.7\% & 2.82\% & 2.0\% & 2.0\% & 1.9\%\\
Net NPA (\%) & 1.1\% & 0.73\% & 0.4\% & 0.3\% & 0.3\%\\
PCR (\%) (excl. AUCA) & 71.6\% & 75.0\% & 80.7\% & 82.7\% & 83.9\%\\
Slippage (\%) & 3.2\% & 2.4\% & 1.8\% & 1.4\% & 1.3\%\\
Credit cost (\%) & 2.8\% & 1.1\% & 1.0\% & 0.3\% & 0.5\%\\
Per share data / Valuation & NaN & NaN & NaN & NaN & NaN\\
EPS (INR) & 21.5 & 42.4 & 31.1 & 84.2 & 100.2\\
BVPS (INR) & 331.8 & 375.2 & 407.6 & 491.8 & 592.0\\
ABVPS (INR) & 308.9 & 357.2 & 396.0 & 480.8 & 580.1\\
P/E (x) & 41.6 & 20.3 & 27.6 & 10.2 & 8.6\\
P/BV (x) & 2.7 & 2.3 & 2.1 & 2.0 & 1.7\\
P/ABV (x) & 2.9 & 2.4 & 2.2 & 2.0 & 1.7
\end{longtblr}

\subsection{RBL Bank}
\begin{enumerate}
    \item In Q4FY23 and FY23, the bank achieved its highest-ever quarterly and annual Profit After Tax (PAT) levels. Additionally, the Return on Assets (RoA) improved by 22 basis points (bps) both year-on-year (YoY) and quarter-on-quarter (QoQ), reaching 1\%. The bank's advances growth for FY23 surpassed initial targets, with a YoY growth of 17\% and a QoQ growth of 5\%. The retail segment witnessed a robust YoY growth of 21\% and a QoQ growth of 8\%. Housing and tractors disbursements accounted for significant portions of the quarterly performance.
    
    \item The core business of the bank includes commercial banking, credit cards (which is expected to continue growing at a rate of 20-25\%), and microfinance. Moreover, the bank plans to include the housing business as part of its core operations within the next six months. Given its relatively low market share (~0.5\%), the bank is positioned for higher-than-industry growth rates.
    
    \item The bank has implemented new practices and processes in the retail segment, and the benefits of these initiatives are expected to materialize in FY24. The bank maintained a strong deposit position, with Current Account and Savings Account (CASA) at 37.4\% and Retail Liquidity Coverage Ratio (LCR) at 42.8\%. Deposits below Rs20 million experienced a YoY growth of 19\% and a QoQ growth of 5\%.
    
    \item In terms of asset quality, the bank observed a decline in Gross Non-Performing Assets (GNPA) and Net Non-Performing Assets (NNPA) in Q4FY23. The Provision Coverage Ratio (PCR) remained flat QoQ at 68\%. Provisions amounted to Rs2.84 billion compared to Rs3.39 billion QoQ. The bank's credit cost for FY24 is estimated to be in the range of 1.5\%-2\%, and it aims to maintain a PCR between 68\% and 70\%. Overall, the bank exhibited stable asset quality and a positive recovery trend.
    
    \item The bank reported gross slippages of Rs6.81 billion, recoveries and upgrades of Rs3.85 billion, and net slippages of Rs2.95 billion. Wholesale gross slippages accounted for Rs1.16 billion, with recoveries and upgrades at Rs1.3 billion, resulting in negative net slippages. The microfinance segment experienced gross slippages of Rs0.71 billion, recoveries and upgrades of Rs0.32 billion, and net slippages of Rs0.39 billion. The cards segment reported gross slippages of Rs2.38 billion, recoveries and upgrades of Rs0.37 billion, and net slippages of Rs2 billion (flat QoQ). Other retail slippages amounted to Rs2.57 billion, with net slippages of Rs0.7 billion (including Rs0.96 billion upgraded from the out-of-order circular of regulators).
    
    \item Looking ahead to FY24-FY26, the bank aims to achieve 20\% growth in advances and deposits. Additionally, an average CASA growth of 1-2\% annually, a rise of 10-20 per annum in RoA, and an increase of 100-150 bps in RoE are targeted. The bank also plans to double its customer count to 26 million, focus on building granularity in the retail deposit franchise, and improve market positions in credit cards and microfinance.
    
    \item Regarding margins, the fixed-to-floating mix is 45:55. The rise in margins over the past three quarters can be attributed to changes in the loan and deposit mix, utilization of excess liquidity, and the lead-lag impact of deposit rates and lending yields. The bank expects stable margins, with any increase in deposit costs likely to be offset by changes in the loan mix.
    
    \item In the cards segment, a change in strategy was implemented for Zomato. The bank witnessed tremendous year-on-year growth in actual spend for FY23 due to the use of credit cards for various non-conventional purposes. Revolve rates are expected to increase by 100-200 bps in FY24. The bank aims for a 20\% growth in granular deposits and remains confident in achieving this goal. Credit card provisions were at 29 bps compared to 39 bps previously. During the quarter, the bank issued 550,000 cards. The credit card portfolio is expected to grow in the range of 20-24\%.
    
    \item In terms of miscellaneous factors, the bank expects a small bump-up in two-wheelers and car segments, while overall operating expenses have been largely reduced for other segments. The bank is comfortable with its current capital buffers and does not anticipate the need to raise capital in the next 18 months.
\end{enumerate}

% \usepackage{tabularray}
\begin{longtblr}[
  caption = {Profit and Loss statement},
]{
  width = \linewidth,
  colspec = {Q[396]Q[87]Q[87]Q[87]Q[94]Q[96]},
  hline{1,21} = {-}{0.08em},
  hline{2} = {-}{0.03em},
}
Year ended 31 Mar (Rs mn) & FY21 & FY22 & FY23 & FY24E & FY25E & \\
Interest income & 82,145 & 81,758 & 91,299 & 1,11,037 & 1,31,032 & \\
Interest expense & -44,270 & -41,491 & -46,784 & -57,742 & -69,586 & \\
Net interest income & 37,876 & 40,267 & 44,515 & 53,294 & 61,446 & \\
growth (\%) & 4.4 & 6.3 & 10.5 & 19.7 & 15.3 & \\
Non-interest income & 18,840 & 23,405 & 24,894 & 29,716 & 35,242 & \\
Operating income & 56,716 & 63,673 & 69,409 & 83,010 & 96,688 & \\
Operating expenses & -27,546 & -36,220 & -47,384 & -55,771 & -65,263 & \\
- Staff expenses & -8,454 & -10,015 & -13,403 & -16,497 & -19,295 & \\
Pre-provisions profit & 29,170 & 27,453 & 22,025 & 27,239 & 31,425 & \\
Core operating profit & 26,449 & 24,961 & 20,425 & 25,539 & 29,625 & \\
growth (\%) & 2.0 & -5.6 & -18.2 & 25.0 & 16.0 & \\
Provisions & -22,279 & -28,604 & -10,219 & -12,698 & -14,398\\
Pre-tax profit \\(before non-recurring items) & 6,891 & -1,151 & 11,805 & 14,541 & 17,027 & \\
Pre-tax profit \\(after non-recurring items) & 6,891 & -1,151 & 11,805 & 14,541 & 17,027 & \\
Tax (current + deferred) & -1,813 & 404 & -2,978 & -3,664 & -4,291 & \\
Net profit & 5,078 & -747 & 8,827 & 10,877 & 12,736 & \\
Adjusted net profit & 5,078 & -747 & 8,827 & 10,877 & 12,736 & \\
growth (\%) & 0.4 & -114.7 & n/a & 23.2 & 17.1 & \\
Net income & 5,078 & -747 & 8,827 & 10,877 & 12,736 & 
\end{longtblr}

% \usepackage{tabularray}
\begin{longtblr}[
  caption = {Balance sheet},
]{
  width = \linewidth,
  colspec = {Q[294]Q[106]Q[106]Q[106]Q[106]Q[106]},
  hline{1,17} = {-}{0.08em},
}
Year ended 31 Mar (Rs mn) & FY21 & FY22 & FY23 & FY24E & FY25E & \\
Investments & 2,32,304 & 2,22,744 & 2,88,755 & 3,10,619 & 3,36,345 & \\
Advances & 5,86,225 & 6,00,218 & 7,02,094 & 8,09,711 & 9,33,160 & \\
Interest earning assets & 9,52,771 & 9,98,439 & 10,76,048 & 12,53,875 & 13,89,482 & \\
Fixed assets (Net block) & 4,665 & 5,481 & 5,740 & 8,545 & 12,465 & \\
Other assets & 49,070 & 58,166 & 76,974 & 1,02,386 & 1,37,517 & \\
Total assets & 10,06,506 & 10,62,086 & 11,58,762 & 13,64,807 & 15,39,464 & \\
Deposits & 7,31,213 & 7,90,065 & 8,48,865 & 9,92,162 & 11,53,547 & \\
Other interest bearing liabilities & 1,12,259 & 1,10,930 & 1,33,313 & 1,20,712 & 1,09,370 & \\
Total Interest bearing liabilities & 8,79,880 & 9,35,903 & 10,22,996 & 12,20,549 & 13,85,263 & \\
Other liabilities and provisions & 36,409 & 34,908 & 40,818 & 1,07,676 & 1,22,346 & \\
Share capital & 5,980 & 5,995 & 5,996 & 5,996 & 5,996 & \\
Reserves  & 1,20,646 & 1,20,187 & 1,29,770 & 1,38,262 & 1,48,205\\
Shareholders' funds & 1,26,626 & 1,26,182 & 1,35,766 & 1,44,258 & 1,54,201 & \\
Total equity & 10,06,506 & 10,62,086 & 11,58,762 & 13,64,807 & 15,39,464
\end{longtblr}


% \usepackage{tabularray}
\begin{longtblr}[
	caption = {Key Ratios},
	label = {tab:key_ratios},
	]{
		width = \linewidth,
		colspec = {Q[475]Q[85]Q[85]Q[85]Q[100]Q[100]},
		cells = {c},
		cell{2}{1} = {c=6}{0.929\linewidth},
		cell{10}{1} = {c=6}{0.929\linewidth},
		cell{21}{1} = {c=6}{0.929\linewidth},
		cell{27}{1} = {c=6}{0.929\linewidth},
		cell{32}{1} = {c=6}{0.929\linewidth},
		cell{36}{1} = {c=6}{0.929\linewidth},
		cell{42}{1} = {c=6}{0.929\linewidth},
		hline{1-3,10-11,21-22,27-28,32-33,36-37,42-43,45} = {-}{},
	}
	& FY21  & FY22  & FY23  & FY24E & FY25E \\
	Valuation Ratios                     &       &       &       &       &       \\
	Adjusted EPS (Rs)                    & 9.3   & -1.3  & 14.7  & 18.1  & 21.2  \\
	BVPS (Rs)                            & 211.7 & 210.5 & 226.4 & 240.6 & 257.2 \\
	Adjusted Book NAV/share (Rs)         & 196.2 & 200.4 & 216.8 & 231.7 & 248.0 \\
	PER (x)                              & 18.5  & NM    & 11.7  & 9.5   & 8.1   \\
	Price/Book (x)                       & 0.8   & 0.8   & 0.8   & 0.7   & 0.7   \\
	Price/Adjusted book (x)              & 0.9   & 0.9   & 0.8   & 0.7   & 0.7   \\
	Dividend Yield (\%)                  & 0.0   & 0.0   & 0.9   & 2.4   & 2.8   \\
	Du-Pont Ratios                       &       &       &       &       &       \\
	NII/Avg. Assets (\%)                 & 4.0   & 3.9   & 4.0   & 4.2   & 4.2   \\
	Non-interest income/Avg Assets (\%)  & 2.0   & 2.3   & 2.2   & 2.4   & 2.4   \\
	- Fee income / Avg Assets (\%)       & 1.7   & 2.0   & 2.1   & 2.2   & 2.3   \\
	- Trading gains / Avg Assets (\%)    & 0.3   & 0.2   & 0.1   & 0.1   & 0.1   \\
	Cost / Avg Assets (\%)               & 2.9   & 3.5   & 4.3   & 4.4   & 4.5   \\
	Non-tax Provisions / Avg Assets (\%) & 2.3   & 2.8   & 0.9   & 1.0   & 1.0   \\
	Tax Provisions / Avg Assets (\%)     & 0.2   & 0.0   & 0.3   & 0.3   & 0.3   \\
	ROA (\%)                             & 0.5   & -0.1  & 0.8   & 0.9   & 0.9   \\
	Leverage                             & 8.2   & 8.2   & 8.5   & 9.0   & 9.7   \\
	ROE (\%)                             & 4.4   & -0.6  & 6.7   & 7.8   & 8.5   \\
	Balance Sheet Ratios                 &       &       &       &       &       \\
	Loan growth (\%)                     & 1.0   & 2.4   & 17.0  & 15.3  & 15.2  \\
	Deposit growth (\%)                  & 26.5  & 8.0   & 7.4   & 16.9  & 16.3  \\
	Loans/Deposits (\%)                  & 80.2  & 76.0  & 82.7  & 81.6  & 80.9  \\
	Investments/Deposits (\%)            & 31.8  & 28.2  & 34.0  & 31.3  & 29.2  \\
	CASA ratio (\%)                      & 31.8  & 35.3  & 37.3  & 36.7  & 36.3  \\
	Profitability Ratios                 &       &       &       &       &       \\
	NIMs (\%)                            & 4.3   & 4.2   & 4.3   & 4.6   & 4.7   \\
	Interest spread (\%)                 & 3.7   & 3.7   & 3.9   & 4.1   & 4.1   \\
	Yield on advances (\%)               & 11.4  & 10.8  & 11.1  & 11.6  & 12.0  \\
	Cost of deposits (\%)                & 5.5   & 4.7   & 5.2   & 5.7   & 5.9   \\
	Efficiency/Other P/L Ratios          &       &       &       &       &       \\
	Non-interest income/Net income (\%)  & 33.2  & 36.8  & 35.9  & 35.8  & 36.4  \\
	Trading income/Net income (\%)       & 4.8   & 3.9   & 2.3   & 2.0   & 1.9   \\
	Cost/Income (\%)                     & 48.6  & 56.9  & 68.3  & 67.2  & 67.5  \\
	Asset Quality Ratios                 &       &       &       &       &       \\
	Gross NPLs (\%)                      & 4.3   & 4.4   & 3.4   & 2.9   & 2.6   \\
	Net NPLs (\%)                        & 2.1   & 1.3   & 1.1   & 0.9   & 0.8   \\
	Net NPLs/Net worth (\%)              & 9.5   & 6.1   & 5.4   & 4.7   & 4.5   \\
	Loan provisions/Avg loans (\%)       & 3.7   & 4.7   & 1.5   & 1.6   & 1.6   \\
	Provisions cover (\%)                & 52.3  & 70.4  & 68.1  & 70.0  & 70.0  \\
	Capitalisation Ratios                &       &       &       &       &       \\
	Tier I Cap. Adequacy (\%)            & 16.6  & 16.2  & 15.3  & 13.6  & 12.7  \\
	Total Cap. Adequacy (\%)             & 17.5  & 16.8  & 16.9  & 15.0  & 13.9  
\end{longtblr}

\subsection{Motherson}

\begin{enumerate}

	\item Samvardhana Motherson (SAMIL) has recently entered into an agreement with Honda Motors to acquire an 81\% stake in Yachiyo Industry's 4W component business. Currently, Honda group owns a 51\% stake in Yachiyo Industry, which is a publicly listed entity. The equity consideration for this acquisition amounts to Euro 145 million, and Yachiyo Industry is a net cash entity. This strategic move is expected to strengthen SAMIL's presence among Japanese Original Equipment Manufacturers (OEMs), particularly Honda, as about 90\% of Yachiyo's revenue comes from Honda itself.

\item Through this acquisition, SAMIL will not only add fuel tanks and sunroofs to its portfolio but also gain opportunities to cross-sell other key products, such as plastic parts, wiring harnesses, and vision systems, across major Japanese OEMs, including Honda. The revenue mix from Honda for SAMIL is anticipated to rise significantly, increasing from 1\% to 6\% after the deal is finalized.

\item Yachiyo Industry's 4W business delivered revenues of Euro 824 million and EBITDA of Euro 92 million in FY23, with an EBITDA margin of approximately 11\%. About half of its revenue is derived from plastic and metallic fuel tanks, while the other half comes from sunroofs, including panoramic sunroofs for premium models. Yachiyo has a significant 9\% share in the global sunroof market and serves various large OEMs globally, extending beyond Honda. The sunroof business is expected to witness growth, driven by increasing sunroof penetration and improved access to clients through SAMIL. Additionally, Yachiyo is actively developing plastic fuel tanks for hydrogen-powered vehicles, which could present further growth opportunities.

\item With Honda accounting for almost 90\% of Yachiyo's revenue and holding a sunroof capacity of 2.5 million units, the deal opens doors for SAMIL to explore sunroof business prospects in India and the EU with more OEMs. Japan constitutes 18\% of Yachiyo's revenue, while China and the US account for 43\% and 33\%, respectively. This acquisition is likely to provide SAMIL with the chance to cross-sell its existing products more effectively in the Japanese market, including to Honda, resulting in a significant boost in revenue share from Honda for SAMIL.

\item The equity consideration for the 81\% stake in Yachiyo's 4W component business stands at Euro 145 million, whereas the FY23 EBITDA was Euro 92 million. The attractive valuation of this deal reflects Honda's willingness to associate with the diversified global component supplier, SAMIL. The deal is expected to enhance SAMIL's consolidated Return on Capital Employed (RoCE) from the current sub-10\% levels. The acquisition is scheduled to be completed in Q1FY25E, taking SAMIL one step closer to its target revenue of around US\$30 billion by FY26, compared to the current level of approximately US\$12 billion.

\item However, there are certain downside risks associated with the deal. A slowdown in the global car market could impact SAMIL's cash flow and pose challenges in funding the deal through debt. Moreover, Yachiyo's past profitability might not be replicated going forward, making the deal valuation appear cheap only in the short term. SAMIL's ability to generate synergistic benefits from Honda and other Japanese OEMs as expected from the deal, and the pressure on the fuel tank business due to the faster adoption of Electric Vehicles (EVs) globally are additional risks to consider.

\end{enumerate}

\subsubsection{Key Financials}
% \usepackage{tabularray}
\begin{longtblr}[
	caption = {Profit and loss statement},
  ]{
	width = \linewidth,
	colspec = {Q[283]Q[96]Q[96]Q[96]Q[96]},
	cell{16}{4} = {c=2}{0.192\linewidth},
	hline{1,22} = {-}{0.08em},
  }
   & FY22 & FY23 & FY24E & FY25E\\
  Net Sales & 6,35,360 & 7,87,007 & 9,06,289 & 9,81,486\\
  Raw material expenditure & 3,67,363 & 4,53,174 & 5,18,186 & 5,69,116\\
  Staff cost & 1,53,746 & 1,79,314 & 2,08,446 & 2,15,927\\
  Other expenses & 69,637 & 92,442 & 99,837 & 1,07,824\\
  Operating expenditure & 5,90,746 & 7,24,929 & 8,26,470 & 8,92,867\\
  EBITDA & 44,614 & 62,077 & 79,819 & 88,619\\
  EBITDA Margin (\%) & 7.0\% & 7.9\% & 8.8\% & 9.0\%\\
  Depreciation & 29,582 & 31,358 & 34,648 & 37,350\\
  EBIT & 15,032 & 30,719 & 45,171 & 51,269\\
  Interest expenditure & 5,426 & 7,809 & 7,000 & 7,000\\
  Non-operating income & 4,957 & 2,570 & 3,084 & 3,701\\
  Adj. PBT & 14,562 & 25,480 & 41,256 & 47,971\\
  Tax & 6,275 & 7,650 & 12,377 & 14,391\\
  Adj. PAT & 8,287 & 17,830 & 28,879 & 33,580\\
  Discontinued PAT & 3,642 & - & - & \\
  Adj. PAT (cont. + discontinued) & 11,929 & 17,830 & 28,879 & 33,580\\
  Minority Interest/Share of JVs & 2,917 & 2,178 & 2,000 & 2,400\\
  Adj. consol PAT after MI & 9,012 & 15,652 & 26,879 & 31,180\\
  Exceptionals & 481 & 995 & - & -\\
  Reported PAT after MI & 8,531 & 14,657 & 26,879 & 31,180
  \end{longtblr}

  % \usepackage{tabularray}
% \usepackage{tabularray}
\begin{longtblr}[
	caption = {Cashflow statement},
  ]{
	width = \linewidth,
	colspec = {Q[488]Q[112]Q[112]Q[112]Q[112]},
	hline{1,16} = {-}{0.08em},
  }
   & FY22 & FY23 & FY24E & FY25E\\
  Operating cashflow before WC changes & 43,181 & 55,864 & 70,527 & 77,930\\
  (Incr) / decr in net working capital & (31,159) & 246 & 10,522 & 824\\
  Cashflow from operations & 12,022 & 56,110 & 81,049 & 78,754\\
  Capex (net) & (47,032) & (50,397) & (31,720) & (34,352)\\
  (Incr) / decrease in investments & - & - & - & -\\
  Cashflow from investments & (47,032) & (50,397) & (31,720) & (34,352)\\
  Net borrowings & 39,347 & (5,952) & - & -\\
  Interest paid & (5,426) & (7,809) & (7,000) & (7,000)\\
  Dividend paid & (2,937) & (4,111) & (8,064) & (8,870)\\
  Others & (7,287) & 10,110 & - & -\\
  Issue of Equity & 1,360 & (1) & - & -\\
  Cashflow from financing & 25,058 & (7,763) & (15,064) & (15,870)\\
  Net change in cash & (9,952) & (2,050) & 34,265 & 28,532\\
  Free cashflow & (35,010) & 5,713 & 49,329 & 44,402
  \end{longtblr}
  % \usepackage{tabularray}
\begin{longtblr}[
  caption = {Balance sheet},
]{
  width = \linewidth,
  colspec = {Q[377]Q[137]Q[137]Q[137]Q[137]},
  hline{1,23} = {-}{0.08em},
}
 & FY22 & FY23 & FY24E & FY25E\\
Shareholders' equity & 4,518 & 6,325 & 6,325 & 6,325\\
Reserves  surplus & 2,01,365 & 2,18,191 & 2,37,006 & 2,59,316\\
Total networth & 2,05,883 & 2,24,516 & 2,43,331 & 2,65,640\\
Minority Interest & 17,763 & 19,254 & 21,254 & 23,654\\
Debt & 1,27,609 & 1,21,657 & 1,21,657 & 1,21,657\\
Deferred tax liability & (8,322) & (8,428) & (8,428) & (8,428)\\
Total liabilities & 3,42,932 & 3,56,998 & 3,77,814 & 4,02,523\\
Gross block & 3,59,032 & 4,07,746 & 4,39,467 & 4,73,819\\
Net block & 2,14,113 & 2,31,469 & 2,28,541 & 2,25,543\\
CWIP & 13,097 & 14,779 & 14,779 & 14,779\\
Investments (non-current) & 64,617 & 62,899 & 62,899 & 62,899\\
Cash  equivalents & 49,994 & 46,987 & 81,252 & 1,09,783\\
Debtors & 80,247 & 98,379 & 99,319 & 1,07,560\\
Inventory & 64,417 & 78,228 & 79,455 & 86,048\\
Loans  advances & 62,449 & 72,133 & 87,179 & 94,390\\
Total current assets & 2,57,107 & 2,95,726 & 3,47,206 & 3,97,782\\
Current liabilities & 1,94,373 & 2,37,248 & 2,60,713 & 2,82,345\\
Provisions & 11,629 & 10,627 & 14,898 & 16,134\\
Total current liabilities & 2,06,002 & 2,47,875 & 2,75,611 & 2,98,479\\
Net current assets & 51,105 & 47,852 & 71,595 & 99,302\\
Total assets & 3,42,932 & 3,56,998 & 3,77,814 & 4,02,523
\end{longtblr}
  % \usepackage{tabularray}
\begin{longtblr}[
  caption = {Key ratios},
]{
  width = \linewidth,
  colspec = {Q[179]Q[58]Q[50]Q[58]Q[58]},
  cell{2}{3} = {c=3}{0.166\linewidth},
  cell{8}{3} = {c=3}{0.166\linewidth},
  cell{15}{3} = {c=3}{0.166\linewidth},
  cell{20}{3} = {c=3}{0.166\linewidth},
  cell{27}{3} = {c=3}{0.166\linewidth},
  hline{1,34} = {-}{0.08em},
}
 & FY22 & FY23 & FY24E & FY25E\\
Per Share Data (in Rs) & , & , &  & \\
EPS (Rs) & 1.4 & 2.5 & 4.2 & 4.9\\
Diluted EPS (Rs) & 1.4 & 2.5 & 4.2 & 4.9\\
CEPS (Rs) & 6.0 & 7.3 & 9.7 & 10.8\\
Dividend per share (Rs) & 0.5 & 0.7 & 1.3 & 1.4\\
Book value per share (Rs) & 32.6 & 35.5 & 38.5 & 42.0\\
Growth Ratios (\%) & , & , &  & \\
Total Op. Income (Sales) & 10.7 & 23.9 & 15.2 & 8.3\\
EBITDA & 2.3 & 39.1 & 28.6 & 11.0\\
Net Income (Adjusted) & 38.7 & 4.1 & 5.8 & 6.5\\
EPS (Adjusted) & (18.5) & 73.7 & 71.7 & 16.0\\
Cash EPS & (4.0) & 20.7 & 33.7 & 11.4\\
BVPS (Adjusted) & 63.9 & 9.1 & 8.4 & 9.2\\
Valuation Ratios (x) & , & , &  & \\
P/E (x) & 59.7 & 34.3 & 20.0 & 17.2\\
P/BV (x) & 2.6 & 2.4 & 2.2 & 2.0\\
EV/Sales (x) & 1.0 & 0.8 & 0.7 & 0.6\\
EV/EBITDA (x) & 13.8 & 9.9 & 7.7 & 6.9\\
Return/Profitability Ratios (\%) & , & , &  & \\
EBITDA Margin & 7.0 & 7.9 & 8.8 & 9.0\\
Net Income Margin (Adjusted) & 1.4 & 2.0 & 3.0 & 3.2\\
RoCE & 3.3 & 6.5 & 8.9 & 9.6\\
RoNW & 4.4 & 7.0 & 11.0 & 11.7\\
Dividend Payout Ratio & 32.6 & 26.3 & 30.0 & 28.4\\
Dividend Yield & 0.5 & 0.8 & 1.5 & 1.6\\
Solvency/Wkg. Cap. Ratios (x) & , & , &  & \\
Net D/E & 0.4 & 0.3 & 0.2 & 0.0\\
Debt/EBITDA & 2.9 & 2.0 & 1.5 & 1.4\\
EBIT/Interest & 2.8 & 3.9 & 6.5 & 7.3\\
Inventory (days) & 33 & 33 & 32 & 32\\
Receivables (days) & 44 & 41 & 40 & 40\\
Payables (days) & 117 & 100 & 105 & 105
\end{longtblr}

\subsection{Camlin Fine Sciences}

\textbf{Est. Vs. Actual for Q4FY23:}
In Q4FY23, the company's revenue was in line with estimates, but EBITDA and PAT missed expectations. Revenue saw a 1.4\% increase compared to estimates and a 10\% year-on-year and quarter-on-quarter growth. However, EBITDA declined significantly by 12\% quarter-on-quarter and missed estimates by 22\% due to a Forex loss, resulting in an EBITDA margin of 10.3\%. The company's PAT stood at Rs 3.48 Cr, down 74\% year-on-year and 85\% quarter-on-quarter, mainly due to an exceptional impairment loss of Patent in the nature of Process Technical Knowhow for the manufacture of Vanillin owned by CFS Wanglong (Chinese Joint Venture).

\textbf{Change in Estimates post Q4FY23:}
Following Q4FY23 results, the company's estimates for FY24E and FY25E have been adjusted. Revenue is expected to grow by 7\% in FY24E and 1\% in FY25E, while EBITDA is projected to decrease by 7\% in FY24E and then recover with a 6\% growth in FY25E. PAT is estimated to decline by 10\% in FY24E and further decrease by 22\% in FY25E.

\textbf{Recommendation Rationale:}
The company's strong growth momentum in Blends and Performance Chemicals is a positive indicator. Additionally, focusing on HQ derivatives (MEHQ) with sales set to begin in the coming quarters adds to its potential. The expectation of CFS Brazil and CFS North America turning EBITDA positive is encouraging, and the Lockheed Martin order, which will be supplied in H2FY24 and onwards, will further contribute to growth. The strengthening of promoter holding to reduce free float can lead to a probable multiple expansion in the future, backed by strong operational performance and improvement in return ratios.

\textbf{Company Outlook \& Guidance:}
The company anticipates monetizing its Vanillin inventory in the upcoming quarters. CFSs have demonstrated strong performance in domestic entities, supported by robust growth in performance chemicals, which is expected to lead to higher earnings in the coming years. The company remains optimistic about global demand for CFS products, as a significant portion (80-85\%) of their products are directly or indirectly used in food consumption. Furthermore, efforts to debottleneck the MEHQ facility and develop new downstream products will enhance the product basket and revenue generation.

\textbf{Current Valuation and Recommendation:}
The company's current valuation stands at 15x FY25E earnings. Despite reducing the FY24/25E EBITDA estimates due to a slower-than-expected ramp-up in the Vanillin facility, the valuation is deemed comfortable at $\sim$12x FY25E earnings. Based on this assessment, the revised target price (TP) is set at Rs 210/share, implying a 25\% upside from the current market price (CMP).

\textbf{Recommendation Rationale \& Key Presentation Highlights:}
The Vanillin Facility commenced production on Jan 22, 2023, but sales from Vanillin have been flattish Q-o-Q. However, the company expects to monetize the Vanillin inventory in Q1 FY24. Going forward, the focus on HQ derivatives (mainly MEHQ) is anticipated to gain momentum as HQ prices normalize globally. CFS Brazil and CFS North America are expected to turn EBITDA positive, with Lockheed Martin order supply starting in Q2-Q3FY24, leading to potential growth in these markets. The recent co-operation agreement between the Promoter \& Managing Director, Mr. Ashish Dandekar, and other stakeholders is expected to strengthen the company's position and benefit from diverse promoter experience.

\textbf{Key Risks to our Estimates and TP:}
Several risks could impact the company's estimates and target price, including global slowdown affecting demand in the coming quarters, any adverse outcomes of the ongoing Russia-Ukraine crisis on international gas prices, and potential delays in approval from customers, leading to slower volume ramp-up of new products.

\subsubsection{Key Financials}

% \usepackage{tabularray}
\begin{longtblr}[
  caption = {Profit \& Loss},
  label = {tab:financial_data},
]{
  width = \linewidth,
  colspec = {Q[467]Q[113]Q[113]Q[121]Q[110]},
  column{even} = {c},
  column{3} = {c},
  column{5} = {c},
  hline{1,22} = {-}{0.08em},
}
\textbf{Y/E March} & \textbf{FY22} & \textbf{FY23} & \textbf{FY24E} & \textbf{FY25E}\\
Total Net Sales & 1,412 & 1,682 & 2,251 & 3,029\\
Sales Growth \% & 19.0\% & 19.1\% & 33.9\% & 34.6\%\\
Total Raw Material Consumption & 759 & 813 & 1,058 & 1,424\\
Staff costs & 145 & 163 & 214 & 288\\
Other Expenditure & 355 & 500 & 630 & 818\\
Total Expenditure & 1,259 & 1,476 & 1,902 & 2,529\\
EBITDA & 153 & 205 & 349 & 500\\
\% Change & -16.0\% & 34.3\% & 70.0\% & 43.3\%\\
EBITDA Margin \% & 10.8\% & 12.2\% & 15.5\% & 16.5\%\\
Depreciation & 56 & 63 & 85 & 93\\
EBIT & 97 & 143 & 263 & 406\\
\% Change & -29.6\% & 47.4\% & 84.4\% & 54.3\%\\
EBIT Margin \% & 6.9\% & 8.5\% & 11.7\% & 13.4\%\\
Interest & 36 & 59 & 71 & 95\\
Other Income & 33 & 6 & 34 & 30\\
PBT & 94 & 80 & 226 & 341\\
Tax & 34 & 41 & 75 & 119\\
Tax Rate \% & 35.8\% & 50.5\% & 33.0\% & 35.0\%\\
PAT & 60 & 40 & 151 & 222\\
PAT Growth \% & -7.6\% & -34.1\% & 280.4\% & 46.3\%
\end{longtblr}

% \usepackage{tabularray}
\begin{longtblr}[
  caption = {Balance Sheet},
  label = {tab:balance_sheet},
]{
  width = \linewidth,
  colspec = {Q[423]Q[125]Q[125]Q[125]Q[125]},
  cells = {c},
  hline{1,30} = {-}{0.08em},
}
\textbf{Y/E March} & \textbf{FY22} & \textbf{FY23} & \textbf{FY24E} & \textbf{FY25E}\\
Share Capital & 16 & 16 & 16 & 16\\
Reserves  Surplus & 732 & 804 & 955 & 1,177\\
Total Shareholders Funds & 765 & 824 & 976 & 1,197\\
Non-Current Liabilities &  &  &  & \\
Long Term Borrowings & 387.6 & 408.1 & 506.4 & 817.8\\
Deferred Tax Liability (Net) & 10.3 & 15.1 & 15.1 & 15.1\\
Total Non-Current Liabilities & 439.8 & 443.0 & 529.3 & 843.3\\
Current Liabilities &  &  &  & \\
Short Term Borrowings & 234.9 & 371.5 & 506.4 & 545.2\\
Trade Payables & 233.8 & 288.4 & 289.8 & 390.0\\
Other Financial Liability & 69.7 & 63.9 & 103.7 & 139.5\\
Other Current Liability & 34.8 & 37.9 & 21.2 & 28.5\\
Total Current Liability & 595.3 & 793.0 & 955.2 & 1,141.9\\
Total Liabilities & 1,035.0 & 1,236.0 & 1,484.5 & 1,985.2\\
Total Equity  Liability & 1,800.1 & 2,060.1 & 2,460.1 & 3,182.3\\
Assets &  &  &  & \\
PPE & 481.8 & 752.5 & 833.6 & 789.9\\
Intangible assets & 76.0 & 66.2 & 76.3 & 76.3\\
Capital Work in Progress & 214.7 & 40.8 & 40.8 & 40.8\\
Total Non-Current Assets & 906.5 & 972.3 & 1,072.9 & 1,037.0\\
Current Assets: &  &  &  & \\
Inventories & 370.9 & 568.1 & 555.0 & 746.9\\
Trade Receivable & 299.7 & 304.6 & 400.8 & 539.4\\
Cash and Cash Equivalents & 107.8 & 93.7 & 156.4 & 476.4\\
Bank Balance & 34.3 & 5.5 & 35.5 & 65.5\\
Other Current Assets & 73.2 & 101.6 & 225.1 & 302.9\\
Total Current Assets & 893.5 & 1,087.8 & 1,387.1 & 2,145.3\\
Total Assets & 1,800.1 & 2,060.1 & 2,460.1 & 3,182.3
\end{longtblr}

% \usepackage{tabularray}
\begin{longtblr}[
  caption = {Cash Flow Statement Data},
  label = {tab:cash_flow},
]{
  width = \linewidth,
  colspec = {Q[540]Q[87]Q[87]Q[108]Q[108]},
  column{even} = {c},
  column{3} = {c},
  column{5} = {c},
  hline{1,22} = {-}{0.08em},
}
\textbf{Y/E March} & \textbf{FY22} & \textbf{FY23} & \textbf{FY24E} & \textbf{FY25E}\\
PBT & 94 & 80 & 226 & 341\\
Depreciation  Amortization & 56 & 63 & 86 & 94\\
Chg in Working cap & 4 & -139 & -209 & -290\\
Direct tax paid & -30 & -20 & -75 & -119\\
Cash flow from operations & 145 & 51 & 65 & 90\\
Chg in Gross Block & -214 & -144 & -102 & -50\\
Chg in Investments & 28 & 29 & 0 & 0\\
Proceeds on redemption of Fin. Assets & 0 & 0 & 0 & 0\\
Cash flow from investing & -249 & -125 & -167 & -27\\
Proceeds / (Repayment) of Short Term & -30 & 52 & 135 & 39\\
Borrowings (Net) &  &  &  & \\
Proceeds from issue of Equity & 1 & 1 & 0 & 0\\
Instruments of the company &  &  &  & \\
Loans & 106 & 81 & 98 & 311\\
Finance Cost paid & -25 & -52 & -71 & -95\\
Dividends paid & -1 & -14 & 0 & 0\\
Cash flow from financing & 136 & 60 & 165 & 257\\
Chg in cash & 32 & -14 & 63 & 320\\
Cash at start & 76 & 108 & 94 & 156\\
Cash at end & 108 & 94 & 156 & 476
\end{longtblr}

% \usepackage{tabularray}
\begin{longtblr}[
  caption = {Financial Ratios and Performance Data},
  label = {tab:financial_ratios},
]{
  width = \linewidth,
  colspec = {Q[385]Q[133]Q[133]Q[140]Q[127]},
  row{2} = {c},
  row{6} = {c},
  row{10} = {c},
  row{16} = {c},
  row{21} = {c},
  row{27} = {c},
  column{even} = {c},
  column{3} = {c},
  column{5} = {c},
  cell{2}{1} = {c=5}{0.918\linewidth},
  cell{6}{1} = {c=5}{0.918\linewidth},
  cell{10}{1} = {c=5}{0.918\linewidth},
  cell{16}{1} = {c=5}{0.918\linewidth},
  cell{21}{1} = {c=5}{0.918\linewidth},
  cell{27}{1} = {c=5}{0.918\linewidth},
  hline{1,29} = {-}{0.08em},
}
\textbf{Y/E March} & \textbf{FY22} & \textbf{FY23} & \textbf{FY24E} & \textbf{FY25E}\\
\textbf{Growth (\%)} &  &  &  & \\
Net Sales & 19.0\% & 19.1\% & 33.9\% & 34.6\%\\
EBITDA & -16.0\% & 34.3\% & 70.0\% & 43.3\%\\
APAT & -7.6\% & -34.1\% & 280.4\% & 46.3\%\\
\textbf{Per Share Data (Rs.)} &  &  &  & \\
Adj. EPS & 3.8 & 2.5 & 9.6 & 14.1\\
BVPS & 48.7 & 52.5 & 62.1 & 76.2\\
DPS & 0.0 & 0.0 & 0.0 & 0.0\\
\textbf{Profitability (\%)} &  &  &  & \\
EBITDA Margin & 10.8\% & 12.2\% & 15.5\% & 16.5\%\\
Adj. PAT Margin & 4.3\% & 2.4\% & 6.7\% & 7.3\%\\
ROCE & 9.0\% & 12.3\% & 20.0\% & 26.6\%\\
ROE & 7.9\% & 4.8\% & 15.5\% & 18.5\%\\
ROIC & 9.0\% & 12.3\% & 20.0\% & 26.6\%\\
\textbf{Valuations (X)} &  &  &  & \\
PER & 43.9 & 66.7 & 17.5 & 12.0\\
P/BV & 3.5 & 3.2 & 2.7 & 2.2\\
EV / EBITDA & 20.7 & 16.3 & 10.1 & 7.1\\
EV / Net Sales & 2.2 & 2.0 & 1.6 & 1.2\\
\textbf{Turnover Days} &  &  &  & \\
Asset Turnover & 2.2 & 1.9 & 2.0 & 2.5\\
Inventory days & 89.3 & 101.9 & 90.0 & 90.0\\
Debtors days & 73.7 & 65.6 & 65.0 & 65.0\\
Creditors days & 94.2 & 117.2 & 100.0 & 100.0\\
Working Capital Days & 68.8 & 50.3 & 55.0 & 55.0\\
\textbf{Gearing Ratio} &  &  &  & \\
Total Debt to Equity (x) & 0.7 & 0.8 & 0.9 & 0.7
\end{longtblr}

\subsection{Ramkrishna Forgings}

Ramkrishna Forgings Ltd (RKFL) is the second-largest forging company in India and is well-positioned to benefit from the resurgence in the Indian automotive sector. Additionally, the company is experiencing increasing demand for forging equipment from non-automotive segments, including railways, oil \& gas, and mining/earthmoving industries. RKFL's growth trajectory is expected to be boosted by timely capacity expansions, diversification of its product mix, exploring export opportunities, and strategic acquisitions and joint ventures.

Over the next few years (FY23-26E), we anticipate RKFL's standalone sales volumes to grow at a Compound Annual Growth Rate (CAGR) of 15.4\%, reaching 2,06,943 tons, compared to the 8.5\% CAGR achieved during FY17-23. This growth projection is expected to drive revenue, EBITDA, and PAT at a CAGR of 15.6\%, 15.8\%, and 26.0\%, respectively, reaching INR 4,635 crore, INR 1,038 crore, and INR 471 crore by FY26E. Furthermore, the company is likely to see an improvement in EBITDA and PAT margins, rising by 14 basis points (bps) to 22.4\% and 232 bps to 10.2\%, respectively, over the same period. These improvements are also anticipated to positively impact the return ratios, with RoE and RoIC expected to rise by 192 bps to 19.7\% and 266 bps to 21.2\%, respectively, by FY26E.

\subsubsection{Key Financials}

% \usepackage{tabularray}
\begin{longtblr}[
  caption = {Income Statement},
]{
  width = \linewidth,
  colspec = {Q[379]Q[110]Q[110]Q[110]Q[110]Q[110]},
  hline{1,32} = {-}{0.08em},
  hline{2,4,6,8,10,12,14,16,18,20,22,24,26,28,30} = {-}{white},
}
\textbf{Y/E March} & \textbf{FY21} & \textbf{FY22} & \textbf{FY23} & \textbf{FY24E} & \textbf{FY25E}\\
\textbf{Revenue} & \textbf{2,285.4} & \textbf{3,001.0} & \textbf{3,548.9} & \textbf{4,090.9} & \textbf{4,634.7}\\
\textit{YoY Growth (\%)} & \textit{77.4} & \textit{31.3} & \textit{18.3} & \textit{15.3} & \textit{13.3}\\
Raw Material Cost & 1,030.8 & 1,443.8 & 1,703.5 & 1,963.6 & 2,224.7\\
\textit{RM Cost to Sales (\%)} & \textit{45.1} & \textit{48.1} & \textit{48.0} & \textit{48.0} & \textit{48.0}\\
Employee Cost & 120.3 & 144.3 & 178.9 & 219.4 & 247.6\\
\textit{Employee Cost to Sales (\%)} & \textit{5.3} & \textit{4.8} & \textit{5.0} & \textit{5.4} & \textit{5.3}\\
Other Expenses & 604.3 & 744.8 & 862.3 & 993.1 & 1,124.2\\
\textit{Other Exp to Sales (\%)} & \textit{26.4} & \textit{24.8} & \textit{24.3} & \textit{24.3} & \textit{24.3}\\
\textbf{EBITDA} & \textbf{530.0} & \textbf{668.2} & \textbf{804.2} & \textbf{914.8} & \textbf{1,038.3}\\
\textit{Margin (\%)} & \textit{23.2} & \textit{22.3} & \textit{22.7} & \textit{22.4} & \textit{22.4}\\
\textit{YoY Growth (\%)} & \textit{130.5} & \textit{26.1} & \textit{20.4} & \textit{13.7} & \textit{13.5}\\
Depreciation  Amortization & 169.1 & 201.4 & 232.4 & 267.1 & 305.3\\
\textbf{EBIT} & \textbf{360.9} & \textbf{466.8} & \textbf{571.8} & \textbf{647.7} & \textbf{733.0}\\
\textit{Margin (\%)} & \textit{15.8} & \textit{15.6} & \textit{16.1} & \textit{15.8} & \textit{15.8}\\
\textit{YoY Growth (\%)} & \textit{217.6} & \textit{29.3} & \textit{22.5} & \textit{13.3} & \textit{13.2}\\
Other Income & 2.9 & 3.8 & 2.8 & 4.2 & 3.8\\
Finance Cost & 93.3 & 115.0 & 112.7 & 111.5 & 107.2\\
Interest Coverage (X) & 3.9 & 4.1 & 5.1 & 5.8 & 6.8\\
Exceptional Item & 0.0 & 0.0 & 0.0 & 0.0 & 0.0\\
\textbf{PBT} & \textbf{270.5} & \textbf{355.6} & \textbf{461.9} & \textbf{540.3} & \textbf{629.7}\\
\textit{Margin (\%)} & \textit{11.8} & \textit{11.9} & \textit{13.0} & \textit{13.2} & \textit{13.6}\\
\textit{YoY Growth (\%)} & \textit{551.6} & \textit{31.5} & \textit{29.9} & \textit{17.0} & \textit{16.5}\\
Tax Expense & 59.7 & 120.1 & 161.7 & 136.0 & 158.5\\
\textit{Tax Rate (\%)} & \textit{22.1} & \textit{33.8} & \textit{35.0} & \textit{25.2} & \textit{25.2}\\
\textbf{PAT} & \textbf{210.8} & \textbf{235.6} & \textbf{300.2} & \textbf{404.3} & \textbf{471.2}\\
\textit{Margin (\%)} & \textit{9.2} & \textit{7.9} & \textit{8.5} & \textit{9.9} & \textit{10.2}\\
\textit{YoY Growth (\%)} & \textit{653.4} & \textit{11.8} & \textit{27.4} & \textit{34.7} & \textit{16.5}\\
Min Int/Sh of Assoc & 0.0 & 0.0 & 0.0 & 0.0 & 0.0\\
\textbf{Net Profit} & \textbf{210.8} & \textbf{235.6} & \textbf{300.2} & \textbf{404.3} & \textbf{471.2}\\
\textit{Margin (\%)} & \textit{9.2} & \textit{7.9} & \textit{8.5} & \textit{9.9} & \textit{10.2}\\
\textit{YoY Growth (\%)} & \textit{653.4} & \textit{11.8} & \textit{27.4} & \textit{34.7} & \textit{16.5}
\end{longtblr}

% \usepackage{tabularray}
\begin{longtblr}[
  caption = {Balance Sheet},
]{
  width = \linewidth,
  colspec = {Q[435]Q[100]Q[100]Q[100]Q[100]Q[100]},
  hline{1,18} = {-}{0.08em},
  hline{3,5,7,9,11,13,15,17} = {-}{white},
}
\textbf{Y/E March} & \textbf{FY21} & \textbf{FY22} & \textbf{FY23} & \textbf{FY24E} & \textbf{FY25E}\\
Share Capital & 32.0 & 32.0 & 32.0 & 32.0 & 32.0\\
Total Reserves & 1,062.1 & 1,293.0 & 1,572.1 & 1,940.1 & 2,359.4\\
\textbf{Shareholders Fund} & \textbf{1,094.1} & \textbf{1,324.9} & \textbf{1,604.1} & \textbf{1,972.1} & \textbf{2,391.4}\\
Long Term Borrowings & 859.5 & 750.9 & 600.0 & 450.0 & 300.0\\
Deferred Tax Assets / Liabilities & 79.3 & 117.2 & 117.2 & 117.2 & 117.2\\
Other Long Term Liabilities & 38.5 & 64.2 & 75.9 & 87.5 & 99.1\\
Long Term Trade Payables & 0.0 & 0.0 & 0.0 & 0.0 & 0.0\\
Long Term Provisions & 0.0 & 0.0 & 0.0 & 0.0 & 0.0\\
\textbf{Total Liabilities} & \textbf{2,071.3} & \textbf{2,257.2} & \textbf{2,397.2} & \textbf{2,626.7} & \textbf{2,907.7}\\
Net Block & 1,465.6 & 1,655.3 & 1,808.0 & 1,970.2 & 2,138.0\\
Capital Work in Progress & 125.1 & 85.1 & 0.0 & 0.0 & 0.0\\
Intangible assets under developmen & 0.0 & 0.0 & 0.0 & 0.0 & 0.0\\
Non Current Investments & 19.4 & 19.4 & 22.9 & 26.4 & 29.9\\
Long Term Loans  Advances & 65.1 & 1.4 & 1.7 & 1.9 & 2.2\\
Other Non Current Assets & 6.0 & 123.6 & 146.2 & 168.5 & 190.9\\
Net Current Assets & 390.0 & 372.4 & 418.4 & 459.6 & 546.6\\
\textbf{Total Assets} & \textbf{2,071.3} & \textbf{2,257.2} & \textbf{2,397.2} & \textbf{2,626.7} & \textbf{2,907.7}
\end{longtblr}

% \usepackage{tabularray}
\begin{longtblr}[
  caption = {Cash Flow Statement},
]{
  width = \linewidth,
  colspec = {Q[398]Q[106]Q[106]Q[106]Q[106]Q[106]},
  hline{1,19} = {-}{0.08em},
  hline{2,4,6,8,10,12,14,16,18} = {-}{white},
}
\textbf{Y/E March} & \textbf{FY21} & \textbf{FY22} & \textbf{FY23} & \textbf{FY24E} & \textbf{FY25E}\\
PBT & 270.5 & 355.6 & 461.9 & 540.3 & 629.7\\
Adjustments & 255.7 & 384.0 & 336.0 & 369.9 & 402.9\\
Change in Working Capital & (425.2) & 158.2 & (193.2) & (155.9) & (156.4)\\
Less: Tax Paid & (59.7) & (120.1) & (161.7) & (136.0) & (158.5)\\
\textbf{Cash Flow from Operations} & \textbf{41.2} & \textbf{777.8} & \textbf{443.0} & \textbf{618.3} & \textbf{717.7}\\
Net Capital Expenditure & (298.4) & (353.1) & (385.1) & (429.3) & (473.1)\\
Change in Investments & (55.0) & 55.8 & 81.5 & (3.5) & (3.5)\\
\textbf{Cash Flow from Investing} & \textbf{(353.5)} & \textbf{(297.3)} & \textbf{(303.6)} & \textbf{(432.8)} & \textbf{(476.6)}\\
Change in Borrowings & 375.2 & (343.8) & 22.6 & (48.6) & (48.3)\\
Less: Finance Cost & (93.3) & (115.0) & (112.7) & (111.5) & (107.2)\\
Proceeds from Equity & 1.8 & 0.0 & 0.0 & 0.0 & 0.0\\
Buyback of Shares & 0.0 & 0.0 & 0.0 & 0.0 & 0.0\\
Dividend Paid & (4.8) & (11.8) & (21.0) & (36.4) & (51.8)\\
\textbf{Cash flow from Financing} & \textbf{278.8} & \textbf{(470.6)} & \textbf{(111.1)} & \textbf{(196.6)} & \textbf{(207.3)}\\
\textbf{Net Cash Flow} & \textbf{(33.4)} & \textbf{9.9} & \textbf{28.3} & \textbf{(11.0)} & \textbf{33.8}\\
Forex Effect & 0.0 & 0.0 & 0.0 & 0.0 & 0.0\\
Opening Balance of Cash & 66.6 & 31.0 & 42.5 & 70.7 & 59.7\\
\textbf{Closing Balance of Cash} & \textbf{33.2} & \textbf{40.9} & \textbf{70.7} & \textbf{59.7} & \textbf{93.5}
\end{longtblr}

% \usepackage{color}
% \usepackage{tabularray}
\begin{longtblr}[
  caption = {Ratio Analysis},
]{
  width = \linewidth,
  colspec = {Q[475]Q[94]Q[75]Q[92]Q[92]Q[92]},
  hline{1,32} = {-}{0.08em},
  hline{3,5,7,9,11,13,15,17,19,21,23,25,27,29,31} = {-}{white},
}
\textbf{Y/E March} & \textbf{FY21} & \textbf{FY22} & \textbf{FY23} & \textbf{FY24E} & \textbf{FY25E}\\
\textbf{Per share data \& Yields} &  &  &  &  & \\
Adjusted EPS (INR) & 13.2 & 14.7 & 18.8 & 25.3 & 29.5\\
Adjusted Cash EPS (INR) & 23.8 & 27.3 & 33.3 & 42.0 & 48.6\\
Adjusted BVPS (INR) & 68.4 & 82.9 & 100.3 & 123.3 & 149.6\\
Adjusted CFO per share (INR) & 2.6 & 48.6 & 27.7 & 38.7 & 44.9\\
CFO Yield (\%) & 0.7 & 14.0 & 8.0 & 11.1 & 12.9\\
Adjusted FCF per share (INR) & (11.7) & 31.3 & 8.2 & 17.0 & 20.3\\
FCF Yield (\%) & (3.4) & 9.0 & 2.4 & 4.9 & 5.8\\
 &  &  &  &  & \\
\textbf{Solvency Ratio (X)} &  &  &  &  & \\
Total Debt to Equity & 1.3 & 0.9 & 0.8 & 0.6 & 0.5\\
Net Debt to Equity & 1.2 & 0.9 & 0.7 & 0.6 & 0.4\\
Net Debt to EBITDA & 2.6 & 1.8 & 1.5 & 1.3 & 1.0\\
 &  &  &  &  & \\
\textbf{Return Ratios (\%)} &  &  &  &  & \\
Return on Equity & 19.1 & 17.8 & 18.7 & 20.5 & 19.7\\
Return on Capital Employed & 11.0 & 12.1 & 13.0 & 15.2 & 15.4\\
Return on Invested Capital & 14.6 & 18.5 & 20.4 & 20.7 & 21.2\\
 &  &  &  &  & \\
\textbf{Working Capital Ratios} &  &  &  &  & \\
Payable Days (Nos) & 92 & 93 & 90 & 90 & 90\\
Inventory Days (Nos) & 109 & 106 & 105 & 105 & 105\\
Receivable Days (Nos) & 140 & 88 & 90 & 90 & 90\\
Net Working Capital Days (Nos) & 157 & 101 & 105 & 105 & 105\\
Net Working Capital to Sales (\%) & 43.1 & 27.6 & 28.8 & 28.8 & 28.8\\
 &  &  &  &  & \\
\textbf{Valuation (X)} &  &  &  &  & \\
P/E & 26.4 & 23.6 & 18.5 & 13.8 & 11.8\\
P/BV & 5.1 & 4.2 & 3.5 & 2.8 & 2.3\\
EV/EBITDA & 13.1 & 10.1 & 8.4 & 7.3 & 6.4\\
EV/Sales & 3.0 & 2.3 & 1.9 & 1.6 & 1.4
\end{longtblr}

\subsection{Titan Company Limited}

\begin{itemize}
  \item Titan Co. Ltd. manufactures and retails jewelry and watches. The company also produces perfumes for men and women.
  \item In Q4FY23, revenue grew 32.9\% YoY to Rs. 10,360cr aided by sharp growth across emerging businesses and watches \& wearables.
  \item EBITDA stood at Rs. 1,203cr, and EBITDA margin remained flat at 11.6\% (+40bps YoY) impacted by some actuarial calculations in watches and one time cleanup of old inventory in lens. PAT grew 43.1\% YoY to Rs. 730cr.
  \item With strong recovery in business segments and network expansions in both domestic and overseas markets, Titan emerged as a resilient player. After the upward trend in jewelry demand during Akshaya Tritiya, the management has predicted strong demand during the wedding season in May and June. However, concerns like gold price volatility and exhaustion of lower cost diamond inventory, will remain in the short term.
  \item The company recorded a healthy revenue growth of 32.9\% YoY at Rs.10,360cr in Q4FY23. Jewellery division grew 23.5\% YoY over a weak base quarter. Major drivers were the $\sim$15\% YoY growth in buyers (both new and repeat)- and $\sim$8\% YoY growth average tickets size. Growth of high-value studded mix and solitaires improved due to seasonal activation along-with wedding segment growing marginally higher than overall retail sales growth. Watches \& wearables grew 41\% YoY – analog watches at 32\% YoY and wearables at 170\% YoY. For the second consecutive quarter, Fastrack brand outpaced the portfolio brands. Eyecare grew by 22.8\% YoY backed by growing volume and partially by average selling price. Emerging businesses clocked 83.9\% YoY growth, led by fragrances \& fashion accessories (31\% YoY) and Indian dress wear (Taneira; 208\% YoY). EBITDA grew by 38.3\% YoY to Rs. 1,203cr in Q4FY23.
  \item Watches \& wearables achieved the significant milestone Rs. 5,000cr+ of uniform consumer price sales, along with another milestone of 52 new stores, thus crossing 1,000 stores pan-India. This is the fastest quarterly expansion in the division's history.
  \item Titan plans to set up 40+ new stores in the jewellery division in FY24-25.
  \item Titan's subsidiary, CaratLane's revenue grew by 58.5\% YoY fuelled by the gifting campaigns around Valentine's Day. Key category, studded business recorded a robust growth of $\sim$57\% YoY. Titan Engineering and Automation Limited declined by 6.7\% YoY, impacted by slow order deliveries in the automation solutions division ($-$23\% YoY), whereas the manufacturing services division witnessed 33\% YoY growth.
  \item Despite challenging times, Titan achieved landmark milestones in annual retail sales in Q4FY23. After tepid demand in the first half of April, jewellery demand increased in the second half of the month due to Akshaya Tritiya sales. The management is confident of robust demand during the wedding season in May and June. However, several challenges – fluctuating gold prices and unavailability of lower cost diamonds – pose major concerns in the short term. Therefore, we remain cautious and reiterate to HOLD with a target price of Rs. 2,960 based on 55x FY25E adj. EPS.
\end{itemize}

% \usepackage{color}
% \usepackage{tabularray}
\begin{longtblr}[
  caption = {PROFIT \& LOSS},
]{
  width = \linewidth,
  colspec = {Q[300]Q[123]Q[123]Q[123]Q[123]Q[123]},
  hline{1,23} = {-}{0.08em},
  hline{2} = {-}{0.05em},
}
Y.E March (Rs. cr) & FY21A & FY22A & FY23A & FY24E & FY25E\\
Revenue & 21,644 & 28,799 & 40,575 & 44,659 & 52,074\\
\% change & 2.8 & 33.1 & 40.9 & 10.1 & 16.6\\
EBITDA & 1,911 & 3,575 & 5,187 & 6,309 & 7,286\\
\% change & (27.1) & 87.1 & 45.1 & 21.6 & 15.5\\
Depreciation & 376 & 399 & 441 & 513 & 510\\
EBIT & 1,535 & 3,176 & 4,746 & 5,796 & 6,776\\
Interest & 203 & 218 & 300 & 322 & 290\\
Other Income & (5) & (54) & 1 & 1 & 1\\
PBT & 1,327 & 2,904 & 4,447 & 5,474 & 6,488\\
\% change & (36.9) & 118.8 & 53.1 & 23.1 & 18.5\\
Tax & 353 & 706 & 1,173 & 1,416 & 1,678\\
Tax Rate (\%) & 26.6 & 24.3 & 26.4 & 25.9 & 25.9\\
Reported PAT & 973 & 2,173 & 3,250 & 4,029 & 4,775\\
Adj* & - & 54 & - & - & -\\
Adj PAT & 973 & 2,227 & 3,250 & 4,029 & 4,775\\
\% change & (35.2) & 128.9 & 45.9 & 24.0 & 18.5\\
No. of shares (cr) & 88.8 & 88.8 & 88.8 & 88.8 & 88.8\\
Adj EPS (Rs.) & 11.0 & 25.1 & 36.6 & 45.4 & 53.8\\
\% change & (35.6) & 128.9 & 45.9 & 24.0 & 18.5\\
DPS (Rs.) & 4.0 & 7.5 & 10.0 & 13.6 & 16.1\\
CEPS (Rs.) & 15.2 & 29.6 & 41.6 & 51.2 & 59.5
\end{longtblr}

% \usepackage{color}
% \usepackage{tabularray}
\begin{longtblr}[
  caption = {BALANCE SHEET},
]{
  width = \linewidth,
  colspec = {Q[288]Q[129]Q[129]Q[129]Q[129]Q[129]},
  hline{1,23} = {-}{0.08em},
  hline{2} = {-}{0.05em},
}
Y.E March (Rs. cr).1 & FY21A.1 & FY22A.1 & FY23A.1 & FY24E.1 & FY25E.1\\
Cash & 560 & 1,573 & 1,343 & 2,294 & 2,758\\
Accounts Receivable & 366 & 565 & 674 & 670 & 781\\
Inventories & 8,408 & 13,609 & 16,584 & 18,418 & 21,264\\
Other Cur. Assets & 3,863 & 1,707 & 3,806 & 4,301 & 4,960\\
Investments & 43 & 280 & 352 & 387 & 426\\
Gross Fixed Assets & 1,957 & 2,173 & 2,593 & 3,090 & 3,667\\
Net Fixed Assets & 2,133 & 2,191 & 2,628 & 2,619 & 2,694\\
CWIP & 19 & 69 & 133 & 120 & 108\\
Intangible Assets & 379 & 368 & 380 & 385 & 390\\
Def. Tax (Net) & 105 & 187 & 158 & 283 & 336\\
Other Assets & 576 & 645 & 965 & 962 & 960\\
Total Assets & 16,452 & 21,194 & 27,023 & 30,439 & 34,677\\
Current Liabilities & 3,348 & 4,598 & 5,770 & 5,833 & 6,406\\
Provisions & 156 & 198 & 241 & 246 & 251\\
Debt Funds & 5,438 & 7,059 & 9,105 & 9,271 & 9,327\\
Other Liabilities & 8 & 6 & 3 & 14 & 17\\
Equity Capital & 89 & 89 & 89 & 89 & 89\\
Reserves  Surplus & 7,408 & 9,214 & 11,762 & 14,903 & 18,469\\
Shareholder’s Fund & 7,497 & 9,303 & 11,851 & 14,992 & 18,558\\
Minority Interest & 5 & 30 & 53 & 83 & 118\\
Total Liabilities & 16,452 & 21,194 & 27,023 & 30,439 & 34,677
\end{longtblr}

% \usepackage{color}
% \usepackage{tabularray}
\begin{longtblr}[
  caption = {CASH FLOW},
]{
  width = \linewidth,
  colspec = {Q[294]Q[125]Q[125]Q[125]Q[125]Q[125]},
  hline{1,17} = {-}{0.08em},
  hline{2} = {-}{0.05em},
}
Y.E March (Rs. cr) & FY21A & FY22A & FY23A & FY24E & FY25E\\
Net inc. + Depn. & 1,349 & 2,572 & 3,691 & 4,542 & 5,285\\
Non-cash adj. & 2,970 & 1,606 & 857 & (534) & (177)\\
Changes in W.C & (180) & (4,902) & (3,178) & (1,695) & (2,761)\\
C.F. Operation & 4,139 & (724) & 1,370 & 2,313 & 2,347\\
Capital exp. & (96) & (216) & (420) & (496) & (578)\\
Change in inv. & - & (153) & (1) & (108) & (114)\\
Other invest.CF & (2,705) & 1,533 & (1,390) & (35) & (39)\\
C.F – Investment & (2,801) & 1,164 & (1,811) & (640) & (730)\\
Issue of equity & - & - & - & - & -\\
Issue/repay debt & (562) & 342 & 1,677 & 166 & 55\\
Dividends paid & (355) & (355) & (666) & (888) & (1,209)\\
Other finance.CF & (317) & (390) & (554) & - & -\\
C.F – Finance & (1,234) & (403) & 457 & (722) & (1,153)\\
Chg. in cash & 104 & 37 & 16 & 951 & 463\\
Closing cash & 560 & 1,573 & 1,343 & 2,294 & 2,758
\end{longtblr}

% \usepackage{color}
% \usepackage{tabularray}
\begin{longtblr}[
  caption = {Ratio Analysis},
]{
  width = \linewidth,
  colspec = {Q[212]Q[133]Q[133]Q[133]Q[133]Q[133]Q[60]},
  column{even} = {r},
  column{3} = {r},
  column{5} = {r},
  hline{1,17} = {-}{0.08em},
  hline{2} = {1-6}{0.03em},
}
Y.E March & FY21A.1 & FY22A.1 & FY23A.1 & FY24E.1 & FY25E.1 & \\
Profitab. & Return & NaN & NaN & NaN & NaN & NaN\\
EBITDA margin (\%) & 8.80 & 12.40 & 12.80 & 14.10 & 14.00 & \\
EBIT margin (\%) & 7.10 & 11.00 & 11.70 & 13.00 & 13.00 & \\
Net profit mgn.(\%) & 4.50 & 7.50 & 8.00 & 9.00 & 9.20 & \\
ROE (\%) & 13.00 & 23.40 & 27.40 & 26.90 & 25.70 & \\
ROCE (\%) & 11.90 & 19.40 & 22.60 & 23.80 & 24.20 & \\
W.C & Liquidity & NaN & NaN & NaN & NaN & NaN\\
Receivables (days) & 6.20 & 7.20 & 6.10 & 5.50 & 5.50 & \\
Inventory (days) & 187.00 & 229.50 & 199.40 & 204.40 & 200.80 & \\
Payables (days) & 17.50 & 21.80 & 14.60 & 15.00 & 14.60 & \\
Current ratio (x) & 1.70 & 1.70 & 1.70 & 1.90 & 2.10 & \\
Quick ratio (x) & 0.50 & 0.20 & 0.30 & 0.40 & 0.40 & \\
Turnover & Leverage & NaN & NaN & NaN & NaN & NaN\\
Gross asset T.O (x) & 11.30 & 13.90 & 17.00 & 15.70 & 15.40 & \\
Total asset T.O (x) & 1.40 & 1.50 & 1.70 & 1.60 & 1.60 & \\
Int. coverage ratio (x) & 7.6 & 14.6 & 15.8 & 18.0 & 23.4 \\
Adj. debt/equity (x) & 0.7 & 0.8 & 0.8 & 0.6 & 0.5 \\
Valuation & & & & & & \\
EV/Sales (x) & 6.6 & 8.0 & 5.7 & 5.6 & 4.8 \\
EV/EBITDA (x) & 74.9 & 64.5 & 44.6 & 39.8 & 34.4 \\
P/E (x) & 142.1 & 101.1 & 68.7 & 60.5 & 51.1 \\
P/BV (x) & 18.4 & 24.2 & 18.8 & 16.3 & 13.1 \\
\end{longtblr}

\subsection{Havells India}

\begin{itemize}
  \item During their interaction with Havells India (Havells), the management displayed confidence in the healthy growth of its core product categories and an improvement in profit margins. The company has been making sustained investments in various aspects like manufacturing, distribution, and marketing to maintain its competitiveness across all categories. Although there has been a slowdown in demand from the business-to-consumer (B-C) segment, the business-to-business (B-B) segment has been largely compensating for it. However, the weak summer season is expected to impact the Lloyd and fan business in the near term.

  \item Havells has been diligently working on liquidating non-rated fan inventory, which should be completed by Q1FY24. Despite the delayed impact of the weak season, the management anticipates a healthy fan performance for the rest of FY24, supported by a price hike of approximately 5%. Additionally, the real estate sector's positive momentum is expected to continue supporting the demand for Havells' core product portfolio, which includes switchgear, cables, and lighting.

  \item Lloyd, a subsidiary of Havells, has made significant strides in the Room Air Conditioner (RAC) segment and is now among the top three players in the industry. Despite the weak summer season, Lloyd has been gaining market share. However, investments in branding, distribution, and manufacturing have impacted profitability. The management believes that the margin has reached its lowest point and will likely improve going forward, thanks to softening commodity inflation and an increase in premium product mix.

  \item The switchgears segment, which accounts for 13\% of revenue and 36\% of EBIT mix, has performed well over the past two years, benefiting from its positive correlation with the construction and real estate sectors. Although growth rates may moderate due to a larger base, Havells remains confident in the segment's positive outlook. 

  \item In the cables and wires segment (33\% revenue and 34\% EBIT mix), wires have experienced faster growth compared to cables, driven by factors such as real estate tailwinds, brand strength, channel expansion, and a shift from the unorganized to organized market.

  \item The electric consumer durables segment (20\% revenue and 27\% EBIT mix) faced challenges in FY23, primarily due to energy rating changes impacting fans and weak consumer sentiment. However, the medium-term outlook remains encouraging, supported by increasing premiumization mix, distribution expansion, and traction in appliances and water heaters.

  \item The lighting segment (9\% revenue and 16\% EBIT mix) demonstrated strong performance in both consumer and professional lighting, driven by geographical expansion strategies and industry-leading margins. However, optical weakness in revenue performance was attributed to a fall in LED prices.

  \item Looking at Lloyd's performance (20\% revenue and -14\% EBIT mix), the RAC industry faced hyper-competition and commodity price volatility in FY23, leading to lower margins. Despite this, Lloyd managed to gain market share and expects to grow ahead of the industry. Margins are likely to remain under pressure in the near term due to competitive forces and increased operating costs, but the benefits of the new manufacturing plant are expected to contribute positively in the future. Lloyd's other products, such as refrigerators, washing machines, and LED TVs, are also gaining traction and complementing its product portfolio.

  \item Overall, the management maintains a positive outlook for Havells and expects the company to continue its growth trajectory, with margins expected to improve in the coming years.
\end{itemize}


% \usepackage{color}
% \usepackage{tabularray}
\begin{longtblr}[
  caption = {Profit and Loss},
]{
  width = \linewidth,
  colspec = {Q[342]Q[98]Q[98]Q[100]Q[98]Q[98]Q[98]},
  column{even} = {c},
  column{3} = {c},
  column{5} = {c},
  column{7} = {c},
  hline{1-2,23} = {-}{},
}
Year End (March) & FY21 & FY22 & FY23 & FY24E & FY25E & FY26E\\
Net Revenues & 1,04,279 & 1,38,885 & 1,68,684 & 1,88,643 & 2,12,106 & 2,37,244\\
Growth (\%) & 10.6 & 33.2 & 21.5 & 11.8 & 12.4 & 11.9\\
Material Expenses & 64,749 & 93,841 & 1,16,713 & 1,28,278 & 1,44,232 & 1,61,326\\
Employee Expense & 8,853 & 10,147 & 12,617 & 13,626 & 14,989 & 16,487\\
ASP Expense & 1,326 & 2,468 & 4,734 & 4,716 & 5,303 & 5,931\\
Distribution Expense & 3,615 & 4,312 & 5,988 & 6,697 & 7,530 & 8,422\\
Other Expenses & 10,083 & 10,542 & 12,602 & 13,543 & 15,175 & 16,077\\
EBITDA & 15,653 & 17,576 & 16,030 & 21,784 & 24,878 & 29,001\\
EBITDA Growth (\%) & 52.4 & 12.3 & (8.8) & 35.9 & 14.2 & 16.6\\
EBITDA Margin (\%) & 15.0 & 12.7 & 9.5 & 11.5 & 11.7 & 12.2\\
Depreciation & 2,489 & 2,608 & 2,961 & 3,327 & 3,667 & 3,893\\
EBIT & 13,164 & 14,968 & 13,069 & 18,457 & 21,211 & 25,107\\
Other Income (Including EO Items) & 1,878 & 1,604 & 1,770 & 2,432 & 2,932 & 3,508\\
Interest & 726 & 534 & 336 & 200 & 150 & 150\\
PBT & 14,316 & 16,038 & 14,503 & 20,690 & 23,993 & 28,465\\
Total Tax & 3,919 & 4,091 & 3,753 & 5,276 & 6,118 & 7,259\\
RPAT & 10,396 & 11,947 & 10,750 & 15,414 & 17,875 & 21,207\\
Adjusted PAT & 10,396 & 11,947 & 10,750 & 15,414 & 17,875 & 21,207\\
APAT Growth (\%) & 41.8 & 14.9 & (10.0) & 43.4 & 16.0 & 18.6\\
EPS & 16.6 & 19.1 & 17.2 & 24.6 & 28.5 & 33.8\\
EPS Growth (\%) & 41.8 & 14.9 & (10.1) & 43.4 & 16.0 & 18.6
\end{longtblr}

% \usepackage{color}
% \usepackage{tabularray}
\begin{longtblr}[
  caption={Balance sheet},
  label = none,
  entry = none,
]{
  width = \linewidth,
  colspec = {Q[379]Q[87]Q[87]Q[87]Q[87]Q[106]Q[106]},
  hline{1,30} = {-}{0.08em},
  hline{4,6,8,10,12,14,16,18,20,22,24,26,28} = {-}{white},
}
\textbf{Year End (March)} & \textbf{FY21} & \textbf{FY22} & \textbf{FY23} & \textbf{FY24E} & \textbf{FY25E} & \textbf{FY26E}\\
\textbf{SOURCES OF FUNDS} &  &  &  &  &  & \\
Share Capital - Equity & 626 & 626 & 627 & 627 & 627 & 627\\
Reserves & 51,019 & 59,260 & 65,518 & 75,294 & 86,903 & 99,965\\
\textbf{Total Shareholders Funds} & \textbf{51,645} & \textbf{59,886} & \textbf{66,145} & \textbf{75,920} & \textbf{87,530} & \textbf{1,00,592}\\
Long Term Debt & 3,937 & 2,726 & - & - & - & -\\
Short Term Debt & 986 & 1,230 & - & - & - & -\\
\textbf{Total Debt} & \textbf{4,922} & \textbf{3,955} & \textbf{-} & \textbf{-} & \textbf{-} & \textbf{-}\\
Net Deferred Taxes & 3,391 & 3,506 & 3,615 & 3,615 & 3,615 & 3,615\\
Other Non Current Liabilities & 1,741 & 2,640 & 3,349 & 3,349 & 3,349 & 3,349\\
\textbf{TOTAL SOURCES OF FUNDS} & \textbf{61,698} & \textbf{69,988} & \textbf{73,109} & \textbf{82,885} & \textbf{94,494} & \textbf{1,07,556}\\
\textbf{APPLICATION OF FUNDS} &  &  &  &  &  & \\
\textbf{Net Block} & \textbf{18,607} & \textbf{20,213} & \textbf{22,278} & \textbf{24,951} & \textbf{24,284} & \textbf{23,391}\\
\textbf{Goodwill} & \textbf{14,333} & \textbf{14,126} & \textbf{13,958} & \textbf{13,958} & \textbf{13,958} & \textbf{13,958}\\
CWIP & 1,011 & 640 & 1,634 & 1,634 & 1,634 & 1,634\\
Non Current Investments & 16 & 2,743 & 205 & 205 & 205 & 205\\
LT Loans  Advances & 634 & 623 & 1,080 & 1,080 & 1,080 & 1,080\\
Other Non Current Assets & 898 & 1,073 & 2,027 & - & - & -\\
\textbf{Total Non-current Assets} & \textbf{35,499} & \textbf{39,418} & \textbf{41,181} & \textbf{41,827} & \textbf{41,160} & \textbf{40,267}\\
Inventories & 26,199 & 29,681 & 37,085 & 37,729 & 42,421 & 47,449\\
Debtors & 5,636 & 7,675 & 9,713 & 10,863 & 12,214 & 13,661\\
Other Current Assets & 1,552 & 1,378 & 2,920 & 3,059 & 3,221 & 3,395\\
Cash  Equivalents & 19,310 & 26,893 & 20,427 & 32,195 & 43,422 & 56,409\\
\textbf{Total Current Assets} & \textbf{52,698} & \textbf{65,626} & \textbf{70,146} & \textbf{83,845} & \textbf{1,01,278} & \textbf{1,20,914}\\
Creditors & 15,968 & 23,794 & 26,425 & 29,552 & 33,228 & 37,166\\
Other Current Liabilities  Provns & 10,531 & 11,262 & 11,792 & 13,235 & 14,716 & 16,459\\
\textbf{Total Current Liabilities} & \textbf{26,498} & \textbf{35,056} & \textbf{38,217} & \textbf{42,787} & \textbf{47,944} & \textbf{53,625}\\
\textbf{Net Current Assets} & \textbf{26,200} & \textbf{30,570} & \textbf{31,928} & \textbf{41,058} & \textbf{53,334} & \textbf{67,289}\\
\textbf{TOTAL APPLICATION OF FUNDS} & \textbf{61,698} & \textbf{69,988} & \textbf{73,109} & \textbf{82,885} & \textbf{94,494} & \textbf{1,07,556}
\end{longtblr}

% \usepackage{color}
% \usepackage{tabularray}
\begin{longtblr}[
  caption = {Cash Flow},
]{
  width = \linewidth,
  colspec = {Q[371]Q[94]Q[94]Q[94]Q[94]Q[94]Q[94]},
  hline{1,24} = {-}{0.08em},
  hline{4,6,8,10,12,14,16,18,20,22} = {-}{white},
}
\textbf{(Rs mn)} & \textbf{FY21} & \textbf{FY22} & \textbf{FY23} & \textbf{FY24E} & \textbf{FY25E} & \textbf{FY26E}\\
Reported PBT & 14,316 & 16,066 & 14,503 & 20,690 & 23,993 & 28,465\\
Non-operating  EO Items & (120) & 91 & (30) & - & - & -\\
Interest Expenses & (411) & (575) & (898) & (200) & (150) & (150)\\
Depreciation & 2,489 & 2,609 & 2,961 & 3,327 & 3,667 & 3,893\\
Working Capital Change & (6,985) & 3,236 & (6,969) & 2,278 & (1,258) & (1,348)\\
Tax Paid & (2,714) & (4,149) & (3,919) & (5,276) & (6,118) & (7,259)\\
\textbf{OPERATING CASH FLOW ( a )} & \textbf{6,574} & \textbf{17,278} & \textbf{5,647} & \textbf{20,818} & \textbf{20,133} & \textbf{23,602}\\
Capex & (1,536) & (2,528) & (5,855) & (6,000) & (3,000) & (3,000)\\
\textit{Free Cash Flow (FCF)} & 5,039 & 14,751 & (208) & 14,818 & 17,133 & 20,602\\
Investments & (7,296) & (6,051) & 5,004 & (2,000) & (2,000) & (2,000)\\
Non-operating Income & 1,203 & 993 & 1,242 & 2,132 & - & -\\
\textbf{INVESTING CASH FLOW ( b )} & \textbf{(7,629)} & \textbf{(7,586)} & \textbf{391} & \textbf{(5,868)} & \textbf{(5,000)} & \textbf{(5,000)}\\
Debt Issuance/(Repaid) & 4,505 & (973) & (3,937) & 255 & 209 & 380\\
Interest Expenses & (459) & (245) & (70) & 200 & 150 & 150\\
\textit{FCFE} & \textbf{10,002} & \textbf{14,022} & \textbf{(4,075)} & \textbf{14,874} & \textbf{17,192} & \textbf{20,831}\\
Share Capital Issuance & 98 & 311 & 267 & - & - & -\\
Dividend & (1,878) & (4,071) & (4,703) & (5,639) & (6,265) & (8,145)\\
Others & (369) & (494) & (626) & - & - & -\\
\textbf{FINANCING CASH FLOW ( c )} & \textbf{1,898} & \textbf{(5,472)} & \textbf{(9,069)} & \textbf{(5,183)} & \textbf{(5,906)} & \textbf{(7,615)}\\
\textbf{NET CASH FLOW (a+b+c)} & \textbf{843} & \textbf{4,221} & \textbf{(3,031)} & \textbf{9,768} & \textbf{9,227} & \textbf{10,987}\\
EO Items, Others & - & - & - & - & - & -\\
Closing Cash  Equivalents & 16,247 & 25,358 & 18,619 & 28,386 & 37,613 & 48,600
\end{longtblr}

% \usepackage{color}
% \usepackage{tabularray}
\begin{longtblr}[
  caption = {KEY RATIOS},
]{
  width = \linewidth,
  colspec = {Q[473]Q[83]Q[83]Q[87]Q[102]Q[102]},
  hline{1,36} = {-}{0.08em},
  hline{3} = {-}{},
  hline{5,7,9,11,13,15,17,19,21,23,25,27,29,31,33,35} = {-}{white},
}
 & \textbf{FY21} & \textbf{FY22} & \textbf{FY23} & \textbf{FY25E} & \textbf{FY26E}\\
 &  &  &  &  & \\
\textbf{PROFITABILITY (\%)} &  &  &  &  & \\
GPM & 37.9 & 32.4 & 30.8 & 32.0 & 32.0\\
EBITDA Margin (\%) & 15.0 & 12.7 & 9.5 & 11.7 & 12.2\\
EBIT Margin & 12.6 & 10.8 & 7.7 & 10.0 & 10.6\\
APAT Margin & 10.0 & 8.6 & 6.4 & 8.4 & 8.9\\
RoE & 22.0 & 21.4 & 17.1 & 21.9 & 22.5\\
RoIC (or Core RoCE) & 26.5 & 27.4 & 22.2 & 32.3 & 36.8\\
RoCE & 19.9 & 18.7 & 15.4 & 20.3 & 21.1\\
\textbf{EFFICIENCY} &  &  &  &  & \\
Tax Rate (\%) & 27.4 & 25.5 & 25.9 & 25.5 & 25.5\\
Fixed Asset Turnover (x) & 3.3 & 3.9 & 4.1 & 4.2 & 4.5\\
\textit{Inventory (days)} & 91.7 & 78.0 & 80.2 & 73.0 & 73.0\\
\textit{Debtors (days)} & 19.7 & 20.2 & 21.0 & 21.0 & 21.0\\
\textit{Other Current Assets (days)} & 5.4 & 3.6 & 6.3 & 5.5 & 5.2\\
\textit{Payables (days)} & 55.9 & 62.5 & 57.2 & 57.2 & 57.2\\
\textit{Other Current Liab \& Provns (days)} & 36.9 & 29.6 & 25.7 & 25.3 & 25.3\\
Cash Conversion Cycle (days) & 24.1 & 9.6 & 24.7 & 17.1 & 16.7\\
Net D/E (x) & (0.2) & (0.4) & (0.3) & (0.4) & (0.5)\\
Interest Coverage (x) & 20.7 & 31.0 & 44.1 & 161.0 & 190.8\\
\textbf{PER SHARE DATA (Rs)} &  &  &  &  & \\
EPS & 16.6 & 19.1 & 17.2 & 28.5 & 33.8\\
CEPS & 20.6 & 23.2 & 21.9 & 34.4 & 40.1\\
Dividend & 6.5 & 7.5 & 7.5 & 11.5 & 13.5\\
Book Value & 82.5 & 95.6 & 105.6 & 139.7 & 160.6\\
\textbf{VALUATION} &  &  &  &  & \\
P/E (x) & 81.1 & 70.6 & 78.5 & 47.2 & 39.8\\
P/BV (x) & 16.3 & 14.1 & 12.8 & 9.6 & 8.4\\
EV/EBITDA (x) & 53.0 & 46.7 & 51.4 & 32.2 & 27.2\\
EV/Revenues (x) & 7.9 & 5.9 & 4.9 & 3.8 & 3.3\\
OCF/EV (\%) & 0.8 & 2.1 & 0.7 & 2.5 & 3.0\\
FCF/EV (\%) & 0.6 & 1.8 & (0.0) & 2.1 & 2.6\\
FCFE/Mkt Cap (\%) & 1.2 & 1.7 & (0.5) & 2.0 & 2.5\\
Dividend Yield (\%) & 0.5 & 0.6 & 0.6 & 0.9 & 1.0
\end{longtblr}