\chapter{Investment Analysis}

\section{Asset Classes}

% \usepackage{tabularray}
\begin{longtblr}[
  label = none,
  entry = none,
]{
  width = \linewidth,
  colspec = {Q[237]Q[77]Q[90]Q[219]Q[312]},
  column{even} = {c},
  column{3} = {c},
  hlines,
  vlines,
}
\textbf{Asset Class} & \textbf{Risk} & \textbf{Return} & \textbf{Benchmarking Agencies} & \textbf{Top Performing Companies}\\
Stocks & High & High & SP 500, Dow Jones, NSE, BSE & Apple, Amazon, Microsoft\\
Bonds & Low & Low to Med & Bloomberg Barclays, Moody's & U.S. Treasury, Goldman Sachs\\
Mutual Funds & Varies & Varies & Morningstar, Lipper, CRISIL & Vanguard, Fidelity, BlackRock\\
Real Estate & Med & Med to High & NCREIF, Zillow, Realtor.com & Brookfield Asset Mgmt, Prologis\\
Commodities & High & High & SP GSCI, Bloomberg Commodity & Glencore, Cargill, Barrick Gold\\
Gold & Med & Med to High & LBMA, COMEX & Barrick Gold, Newmont, AngloGold Ashanti\\
Cryptocurrencies & Very High & Very High & CoinMarketCap, CoinGecko & Bitcoin, Ethereum, Binance Coin\\
Government Bonds & Low & Low to Med & U.S. Treasury, Eurostat & U.S. Treasury Bonds, German Bunds\\
Corporate Bonds & Med & Low to Med & Moody's, SP, Fitch & Apple, Microsoft, Verizon\\
Treasury Bills & Low & Low & U.S. Treasury & U.S. Treasury Bills\\
Exchange-Traded Funds (ETFs) & Varies & Varies & Morningstar, SP, NASDAQ & SPDR SP 500 ETF, Invesco QQQ Trust\\
Venture Capital & Very High & High & Crunchbase, PitchBook & Sequoia Capital, Andreessen Horowitz\\
Private Equity & Very High & High & Preqin, PitchBook & Blackstone Group, KKR  Co.\\
Hedge Funds & Very High & High & Hedge Fund Research (HFR) & Bridgewater Associates, Renaissance Technologies\\
Derivatives & High & High & Chicago Mercantile Exchange & CME Group, Chicago Board Options Exchange\\
Options & High & High & Chicago Board Options Exchange & Apple, Microsoft, Amazon\\
Futures & High & High & Chicago Mercantile Exchange & Crude Oil, Gold, SP 500 Index\\
REITs (Real Estate Investment Trusts) & Med & Med to High & FTSE Nareit, MSCI & Prologis, Simon Property Group\\
Peer-to-Peer Lending & High & High & LendingClub, Prosper & LendingClub, Prosper\\
Art and Collectibles & Med & Med to High & Sotheby's, Christie's & Leonardo da Vinci, Picasso
\end{longtblr}


\section{Key Insights from Survey}

\begin{enumerate}
  \item \textbf{Age and Education Level Distribution:} The majority of survey participants fall into the age group of 26-35 (40\%), followed by the age group of 18-25 (30\%). The education level is relatively high, with 55\% holding a Bachelor's degree and 30\% having a Master's degree or above. This indicates that the survey captures a well-educated and diverse group of participants.
  
  \item \textbf{Investment Experience:} The survey participants have varying levels of investment experience, with the majority (65\%) having 1-10 years of experience. It's interesting to note that 10\% of respondents have less than one year of investment experience, showing a presence of new and inexperienced investors in the sample.
  
  \item \textbf{Types of Investments Made:} Stocks are the most popular investment choice among participants, with 70\% investing in them, followed by bonds (45\%) and mutual funds (55\%). Cryptocurrencies also seem to be gaining popularity, with 25\% of participants investing in them.
  
  \item \textbf{Risk Tolerance:} The survey reveals that 40\% of participants consider themselves to be moderate risk-takers (scoring 3 on a scale of 1 to 5). Additionally, there is a significant portion (25\%) of risk-averse investors (scoring 2), and only a small percentage (5\%) are classified as very risk-takers (scoring 5). This shows a diverse range of risk tolerance levels among the respondents.
  
  \item \textbf{Decision-Making Style:} The majority of participants (40\%) describe their decision-making style as analytical, followed by intuitive (30\%) and impulsive (15\%). This indicates that a significant portion of investors base their decisions on thorough analysis and research.
  
  \item \textbf{Sources of Investment Information:} Financial news websites and financial advisors are the top sources of investment information, with 60\% and 45\% of participants using them, respectively. Social media (35\%) and family and friends (20\%) also play a role in influencing investment decisions.
  
  \item \textbf{Investment Goals:} Capital appreciation is the most common investment goal among respondents, with 65\% choosing it. Wealth preservation (50\%) and retirement planning (40\%) are also important objectives for the participants.
  
  \item \textbf{Preferred Investment Duration:} Medium-term investments (3-10 years) are the most preferred among the participants, with 50\% choosing this duration. Long-term investments (10+ years) are favored by 35\% of respondents.
  
  \item \textbf{Factors Influencing Investment Decisions:} Historical performance is the most crucial factor influencing investment decisions, with 70\% of participants considering it. The risk-reward ratio (60\%) and expert recommendations (40\%) also play significant roles in decision-making.
\end{enumerate}

Overall, the survey provides valuable insights into the diverse preferences and decision-making patterns of investors. It highlights the importance of considering risk tolerance, investment goals, and decision-making styles when providing investment advice or designing financial products tailored to different segments of investors. Additionally, the survey shows the increasing popularity of certain investment options, such as cryptocurrencies, and the influence of various sources of information in shaping investment decisions.

% \newpage
\section{Investment Portfolios}

To design investment portfolios for individuals with varying annual earnings, we'll consider a range of portfolio options based on different risk profiles and investment goals. It's important to note that the allocation percentages provided are general guidelines and should be adjusted based on individual risk tolerance, investment knowledge, and time horizon. Here are six portfolio options across different earning classes:

\begin{itemize}
  \item Portfolio for Earnings of 10 Lakhs:
  \begin{itemize}
    \item Balanced Portfolio:
    \begin{itemize}
      \item Equity: 50\%
      \item Bonds: 40\%
      \item Cash: 10\%
      \item Reasoning: This portfolio aims to strike a balance between growth and stability. With a higher allocation to equities, it offers potential for long-term capital appreciation, while bonds and cash provide stability and income generation.
    \end{itemize}
    \item Conservative Portfolio:
    \begin{itemize}
      \item Equity: 30\%
      \item Bonds: 60\%
      \item Cash: 10\%
      \item Reasoning: This portfolio prioritizes capital preservation and income generation. The higher allocation to bonds provides a more conservative investment approach, with reduced exposure to market volatility.
    \end{itemize}
  \end{itemize}
  
  \item Portfolio for Earnings of 15 Lakhs:
  \begin{itemize}
    \item Growth Portfolio:
    \begin{itemize}
      \item Equity: 70\%
      \item Bonds: 25\%
      \item Cash: 5\%
      \item Reasoning: With a higher allocation to equities, this portfolio is designed for long-term growth. It aims to capitalize on potential market appreciation while maintaining a modest allocation to bonds for diversification and stability.
    \end{itemize}
    \item Moderate Portfolio:
    \begin{itemize}
      \item Equity: 50\%
      \item Bonds: 40\%
      \item Cash: 10\%
      \item Reasoning: This portfolio strikes a balance between growth and stability. It allows for potential capital appreciation through equity investments while maintaining a significant allocation to bonds for income generation and risk reduction.
    \end{itemize}
  \end{itemize}
  
  \item Portfolio for Earnings of 20 Lakhs:
  \begin{itemize}
    \item Aggressive Growth Portfolio:
    \begin{itemize}
      \item Equity: 80\%
      \item Bonds: 15\%
      \item Cash: 5\%
      \item Reasoning: This portfolio focuses on long-term growth and capital appreciation. With a higher allocation to equities, it assumes a higher level of risk but aims to capture market opportunities for potentially higher returns.
    \end{itemize}
    \item Income and Growth Portfolio:
    \begin{itemize}
      \item Equity: 60\%
      \item Bonds: 30\%
      \item Cash: 10\%
      \item Reasoning: This portfolio balances income generation and growth. The allocation to bonds provides stability and income, while equities offer potential for capital appreciation over the long term.
    \end{itemize}
  \end{itemize}
  
  \item Portfolio for Earnings of 30 Lakhs:
  \begin{itemize}
    \item Growth and Income Portfolio:
    \begin{itemize}
      \item Equity: 60\%
      \item Bonds: 25\%
      \item Real Estate: 10\%
      \item Cash: 5\%
      \item Reasoning: This portfolio focuses on both capital appreciation and income generation. It diversifies across equities, bonds, and real estate to provide a combination of growth potential, stability, and passive income.
    \end{itemize}
    \item Diversified Portfolio:
    \begin{itemize}
      \item Equity: 50\%
      \item Bonds: 30\%
      \item Real Estate: 15\%
      \item Cash: 5\%
      \item Reasoning: This portfolio offers a balanced mix of asset classes for diversification. It aims to provide growth through equities, stability through bonds, and potential income through real estate investments.
    \end{itemize}
  \end{itemize}
  
  \item Portfolio for Earnings of 50 Lakhs:
  \begin{itemize}
    \item Capital Growth Portfolio:
    \begin{itemize}
      \item Equity: 70\%
      \item Bonds: 20\%
      \item Real Estate: 7\%
      \item Cash: 3\%
      \item Reasoning: This portfolio emphasizes long-term capital growth with a higher allocation to equities. The inclusion of real estate provides diversification and potential additional income streams.
    \end{itemize}
    \item Income and Diversification Portfolio:
    \begin{itemize}
      \item Equity: 60\%
      \item Bonds: 20\%
      \item Real Estate: 20\%
      \item Cash: 10\%
      \item Reasoning: This portfolio balances income generation, diversification, and growth. It includes a mix of equities, bonds, and real estate to provide potential for capital appreciation, stability, and income generation.
    \end{itemize}
  \end{itemize}
  
  \item Portfolio for Earnings of 1 Crore:
  \begin{itemize}
    \item High-Growth Portfolio:
    \begin{itemize}
      \item Equity: 80\%
      \item Real Estate: 15\%
      \item Cash: 5\%
      \item Reasoning: This portfolio is designed for long-term capital growth. It focuses on equities and real estate, which have historically offered higher growth potential, while maintaining a small cash allocation for liquidity.
    \end{itemize}
    \item Income and Diversification Portfolio:
    \begin{itemize}
      \item Equity: 50\%
      \item Bonds: 20\%
      \item Real Estate: 20\%
      \item Cash: 10\%
      \item Reasoning: This portfolio allows for customization based on the investor's risk tolerance and investment goals. It offers a mix of asset classes for growth, stability, income, and diversification.
    \end{itemize}
  \end{itemize}
\end{itemize}

